\section{Final Conclusions}


In conclusion, the models described and evaluated in this work have shown varying degrees of success in predicting the instant energy production curve of the plant during gaps. The Multi-Layer Neural Network-based model, despite being computational lightweight and efficient, demonstrated limited ability to generalize and predict production variations effectively. It struggled to capture the possibility of production spikes and had difficulty managing the day/night cycle.

While the Recurrent Neural Network-based model exhibited promise, particularly in capturing diverse energy production variations and handling more complex and variable patterns, it outshone the other model in various ways. This model can effectively manage gaps of various sizes, even surpassing the dimensions it was initially trained on, all without necessitating a repeated training phase. Furthermore, it demonstrates the ability to discern the day/night cycle and adapt production output accordingly.

Overall, the models provide valuable insights into energy production trends, and their predictability can be useful in real photovoltaic implants contexts. However, further model refinement and optimization, along with the introduction of techniques that can handle variable gap sizes and more complex patterns, may be necessary to achieve more accurate and reliable predictions. Additionally, expanding the dataset and applying advanced techniques for time series prediction could contribute to more successful models in the future.