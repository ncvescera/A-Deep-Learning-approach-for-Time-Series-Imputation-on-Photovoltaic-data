\section{Final Conclusions}
In conclusion, the models described and evaluated in this work have shown varying degrees of success in predicting the instant energy production curve of the plant during gaps. The Recurrent Neural Network-based model, despite being lightweight and efficient, demonstrated limited ability to generalize and predict production variations effectively. It struggled to capture the possibility of production spikes and had difficulty managing the day/night cycle.

While the Unified Fully Connected Network model showed promise, especially in capturing some of the energy production variations, it, too, had limitations when faced with more complex and variable patterns. Both models displayed difficulties in handling gaps of differing sizes.

Overall, the models provide valuable insights into energy production trends, and their predictability can be useful in certain contexts. However, further model refinement and optimization, along with the introduction of techniques that can handle variable gap sizes and more complex patterns, may be necessary to achieve more accurate and reliable predictions. Additionally, expanding the dataset and applying advanced techniques for time series prediction could contribute to more successful models in the future.