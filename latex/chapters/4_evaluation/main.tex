\chapter{Experimental Results}
In this final chapter, we will analyze the results of the
models previously described by comparing their respective
test phases, highlighting any issues and strengths of the model.
We will test their performance using the Test Set,
as described in Section~\ref{sec:datasetsplitting}, and
evaluate them using different metrics.
Among these metrics, we will employ MAE (Mean Absolute Error) and MAPE (Mean Absolute Percentage Error) \cite{metrics}
to highlight the difference between the model's predictions and
the ground truth and the $R^2$ (R-squared) \cite{metrics} index to understand how well
the model-predicted curve approximates the real one.
Here's a brief introduction to the indices we will use:

%In questo ultimo capitolo andremo ad analizzare i risultati dei modelli
%precedentemente descritti confrontando le rispettive fasi di training,
%mettendo in evidenza eventuali problemi e punti di forza del modello.
%Testeremo le perfoance di questi utilizzando il dataset di Testing, descritto
%nella Sezione~\ref{sec:datasetsplitting}, valutandoli impiegando alcune
%metriche come il MAE (Mean Absolute Error) per evidenziare la diferenza tra predizione del modello e la ground turth, e l'indice $R^2$ per capire
%quanto la curva predetta dal modello riesca bene ad approssimare quella reale.
%Di seguito una breve introduzione agli indici che utilizzeremo:

\begin{itemize}
	\item \textbf{Mean Absolute Error (MAE)}: This metric represents the average pointwise error that the model makes compared to the ground truth. It is expressed in kW.
	      %MAE dell'intero buco, indica in meda l'errore puntuale che il modello commette rispetto alla ground truth. Viene espresso in kW.
	\item \textbf{Mean Absolute Percentage Error (MAPE)} \cite{metrics}: It indicates the error of the model relative to the ground truth, measured in percentage. The calculation of this metric in our case can be quite problematic due to the presence of many values close to zero or exactly zero. Therefore, the calculation is performed by only considering non zero values. Specifically, in addition to the standard MAPE, we will employ a refined version named as MAPE@$k$. In this variant, the metric is calculated exclusively for values in the ground truth greater than $k$. This choice is driven by the distortion introduced in MAPE values when incorrect predictions are linked to the lowest values. Such predictions might exhibit low absolute errors, yet correspond to significantly high percentage errors.
	      %MAPE, indica in percentuale quanto è l'errore del modello rispetto la groundh truth, misurato in percentuale
	\item \textbf{$R^2$ (R-squared)}: This index quantifies how well the prediction curve approximates the ground truth. It helps us understand how well the model approximates the trend of the reference instant prediction. This varies between 0.0, the worst case, to an ideal value of 1.0 \cite{metrics}.
	      %({\bf (*VP* aggiungerei il range di variazione dell'indice, così è chiaro chi sono massimo e minimo raggiungibili.)}
	      %$R^2$, questo indice ci va a quantificare quanto bene la curva della predizione approssimi quella della ground truth. Ci permette di
	      %capire quanto bene il modello riesce ad approssimare l'andamento della predizione istantanea di riferimento.
	      %\item \textbf{Daily MAE} FORSE TOGLIERE: It is calculated only for individual days within the gap. This metric highlights the average daily error that the model makes, measured in kW.
	      %MAE giornaliero, viene calcolato solo sui singoli giorni del buco. Mette in evidenza l'errore giornaliero medio che commette il modello misurato in kW.
	      %\item \textbf{Global MAE}: This is the average of all daily errors, expressed in kW.
	      %Global MAE , è la media di tutti i gli errori giornalieri espresso in kW.

\end{itemize}

\section{Testing Procedure}
All models will be tested using the same test set  introduced in the previous Chapter~\ref{chap:datapreprocessing}. It's important to note that this dataset consists of 33 features, which have been carefully selected as described in Section~\ref{sec:featureselection}, and it contains all the data sampled at 15-minute intervals during the month of April (2880 records). Each model will be tested at least once on this set, according to its suitable input format and capabilities.

%Furthermore, the RNN and Transformer models will undergo additional testing with this set, but the gap size will be fixed at 2 days. This choice was made to facilitate comparisons with the MLP model, which has a rigid input structure. If necessary, the latter two models may also undergo further testing with variations or fixed gap sizes.
%{\bf (*VP* questo paragagrafo è scritto male... e mi pare abbastanza inutile visti i seguenti. )}

The testing procedure applied to the MLP and RNN models for the purpose of comparison (with a fixed 2-day gap size) will be conducted as follows: a 4-day time window will be applied to the testing dataset, shifting forward by 1 day at each iteration. This approach will generate a 2-day gap to predict, along with the corresponding day before and after. For example, consider the time interval from April 2nd to April 5th (time window). In this case, the gap to predict includes all the data of April 3rd and April 4th, while April 2nd and April 5th are used, respectively as the data before and after the gap.

For the Transformer model, the testing procedure will be similar, with the only difference being the size of the time window. In the case of the Transformer, a 1-week time window will be used, and it will shift forward by 1 day at each iteration due to the model's input format requirements. This approach will ensure that a 2-day gap with relevant data before and after is available for the Transformer model as well.

%Tutti i modelli verranno testati con lo stesso testing set (Aprile 2023) introdotto nel precedente Capitolo~\ref{chap:datapreprocessing}. Ricordiamo come questo è composto da 33 features, che sono state accuratamente selezionate come descritto in sezione~\ref{sec:featureselection}, e contine dati campionati ad un itervallo di 15 minuti. Ogni modello verrà testato almeno una volta su questo set nelle modalità più consone e secondo l'input richiesto ed in base alle sue capacità. Poi i modelli RNN e Transformer verranno ulteriormente testati, sempre con questo set, andando però a fissare la lunghezza dei buchi creati a 2 giorni in moto tale da poter essere confrontati successivamente con MLP data la sua rigidità nell'input. Qual'ora fosse necessario, questi ultimi due, potrebbero essere testati ulteriormente andando a variare o fissare la gap size.
%
%La procedura di testing che verrà applicata ai modelli MLP e RNN per essere poi confrontati (gap size fissa a 2 giorni) si svolgerà nel seguente modo: avremo una finestra temporale di 4 giorni che verrà applicata al dataset di testing e scorrerà sempre di 1 giorno in avanti. In questo modo potremmo ottenere, per tutti i modelli, un buco di 2 giorni e il relativo giorno prima e dopo. Per esempio, prendiamo l'intervallo di tempo che va dal 2 Aprile fino al 5 Aprile (finestra temporale). In questo caso il buco da predirre sarà relativo ai giorni 3 e 4, mentre avremo il 2 e il 5 come prima e dopo.
%Mentre per il transformer cambierà solo la dimensione della finestra temporale, che sarà di una settimana e si sposterà in avanti di 1 giorno, per via del formato dell'input che richiede il modello.
%\newpage


\section{MLP}\label{sec:mlpbaseline}
In this section, we will introduce the first model, which we will refer to as the
baseline model, based on a Multi-Layer Perceptron.
We will analyze its architecture, the training phase, and its final performance.

%In questa sezione andremo ad introdurre il primo modello a cui faremo
%riferiento come baseline model, basato su MultiLayer Perceptron.
%Analizzeremo la sua architettura, la 
%fase di training e le sue performance finali.

\subsection{Architecture}
This neural network is designed to predict the instantaneous energy output over
a period of exactly 2 days.
To do this, it requires input data that includes the performance of the system
(features selected during the Preprocessing phase, see Chapter \ref{chap:datapreprocessing}) for exactly one day
before and one day after the period in question.
This enables it to understand how the system is performing and,
consequently, provide the energy output trend.
The model consists of 6 main layers:

%Questa rete neurale è progettata per prevedere l'andamento dell'energia istantanea prodotta durante un periodo di un buco di esattamente 2 giorni.
%Per farlo ha bisogno di avere come input l'andamento dell'impianto (features selezionate nella fase di preprocessing) di esattamente un giorno prima del
%buco e di un giorno dopo. In questo modo può riuscire a comprendere come l'impianto sta performando e quindi restituire l'andamento dell'energia.
%Il modello è formato da 6 livelli principali:

\begin{figure}[H]
	\begin{minipage}{0.6\textwidth}
		\begin{itemize}
			\item \textbf{Input layer}: This is the first layer of the network.
			      It takes two input tensors: \textit{before} and \textit{after}.
			      These tensors represent the day before and the day after the specific period we want to
			      predict. They have the shape \verb|[BATCH_SIZE, 96, 33]|, where 96 is the number
			      of timestamps in our dataset that make up one day, and 33 represents the
			      features obtained from the Data Preprocessing phase.
			      These tensors are then flattened and concatenated to be passed to the subsequent layer.
			      The output of this layer goes through a Batch Normalization layer, and the
			      Rectified Linear Unit (ReLU) activation function is used.

			      % è il primo livello della rete, prende in input 2 tensori: \textit{before} e \textit{after}. Questi stanno ad indicare rispettivamente il giorno prima e quello dopo del buco che vogliamo chiudere. Avranno la forma \verb|[BATCH_SIZE, 96, 33]|, dove \verb|96| è il numero di timestamp del nostro dataset che formano un giorno e \verb|33| sono le feature ottenute dalla fase di preprocessing dei dati. A questi verrà poi applicata un'operazione di Flatten ed infine concatenati per poter essere poi passati al layer successio.
			      %L'output di questo layer passa per un livello di Batch Normalization e viene utilizzata la ReLu come funzione di attivazione. 

			\item \textbf{Hidden layers}: In total, there are 4 layers, each of which takes the
			      output of the previous layer as input and reduces the number of neurons by half.
			      Batch Normalization is applied to the result, and the Rectified Linear Unit (ReLU)
			      is used as the activation function.

			      %in totale 4, ognuno prende in input il risultato del layer precedente e va a dimezzare il numero dei neuroni. Al risultato viene applicata un'operazione di Batch Normalization e utilizzata la ReLu come funzione di attivazione.

			\item \textbf{Output layer}: The final layer of our network, it takes the result from the
			      previous layer and outputs the value of the Instantaneous Energy Produced
			      during the specific period.
			      It produces a tensor with a shape of \verb|[BATCH_SIZE, 192, 1]|.
			      The SoftPlus function is used as the final activation function.

			      %ultimo layer della nostra rete, dal risultato del layer precedente riporta come output il valore dell'Energia Istantanea Prodotta durante il buco, un tensore di forma \verb|[BATCH_SIZE, 192, 1]|. Viene utilizzata la SoftPlus come funzione di attivazione finale.
		\end{itemize}
	\end{minipage}%
	\hspace{0.5cm}
	\begin{minipage}{0.4\textwidth}
		\centering
		\includegraphics[width=0.5\textwidth]{chapters/3_models/imgs/ufcnmodel.png}
		\caption{Beseline Model architecture visualization.}\label{fig:baselinemodelarch}
	\end{minipage}
\end{figure}

\begin{minipage}[t]{0.5\textwidth}
	\begin{figure}[H]
		\centering
		\begin{tikzpicture}
			\draw[->] (-3,0) -- (3,0) node[right] {$x$};
			\draw[->] (0,-1) -- (0,4) node[above] {$y$};
			\draw[dotted] (-3,-1) grid (3,4);
			\draw[color=blue, domain=-3:0] plot[id=logistic] function{0};
			\draw[color=blue, domain=0:3] plot[id=logistic] function{x};
		\end{tikzpicture}
		\caption{Rectified Linear Unit function. $ReLu(x) = \max(0, x)$}
		\label{fig:relu}
	\end{figure}
\end{minipage}%
\hspace{.5cm}
\begin{minipage}[t]{0.5\textwidth}
	\begin{figure}[H]
		\centering
		\begin{tikzpicture}
			\draw[->] (-3,0) -- (3,0) node[right] {$x$};
			\draw[->] (0,-1) -- (0,4) node[above] {$y$};
			\draw[dotted] (-3,-1) grid (3,4);
			\draw[color=blue, domain=-3:3] plot[id=logistic] function{log(1+exp(x))};
		\end{tikzpicture}
		\caption{SoftPlus function. $SoftPlus(x) = \frac{1}{\beta} \log(1+e^{\beta x})$}
		\label{fig:softplus}
	\end{figure}
\end{minipage}

\subsection{Training}
The model was trained by artificially creating gaps of two days in length within the
training dataset.
These gaps were passed to the network in the format described earlier,
and the output result was compared to the actual instantaneous energy produced during the gap.
An \textit{Early Stopping} procedure was implemented to prevent training from continuing if
the model was not improving its performance.
A procedure, called \textit{Save Best}, has also been integrated,
which saves the model to a file whenever the Validation Loss improves.
The validation dataset was applied in this phase to assess the learning progress at the end
of each epoch.
A normalization procedure of the area was applied to the model's output in relation to
that of the gap to ensure that the output starts exactly from the last value of the
instantaneous energy produced \textit{before} and ends exactly with the first value
of the energy produced \textit{after}. Adam was used as the optimizer,
and L1Loss (Equation~\ref{eq:l1loss}) served as the loss function.
We chose to set the batch size parameter to 10, the learning rate $\lambda$ to 0.01,
a maximum of 100 epochs, and a patience value of 20 for Early Stopping.

%Il modello è stato allenato creando, in modo artificiale, buchi di lunghezza
%pari a due giorni all'interno del dataset di training, passati alla rete nel formato
%descritto precedentemente e confrontato il risultato in output con l'effettiva energia istantanea
%prodotta del buco. \'{E} stata implementata una procedura di \textit{Early Stopping} per evitare
%di continuare l'allenamento anche se il modello non sta migliorando le sue performance. 
%\'{E} stata anche integrata una procedura, chiamata \textit{Save Best},
%che salva il modello su file ogni qualvolta la validation loss migliori.
%Il dataset di validation è stato applicato in questa fase per verificare lo stato di
%apprendimento alla fine di ogni epoca. All'output del modello è stata applicata una 
%procedura di normalizzazione dell'area rispetto a quella del buco per far si che la predizione
%parta esattamente dall'ultimo valore dell' energia istantanea prodotta di \textit{before} e
%termini esattamente con il primo valore di \textit{after}.
%\'{E} stato impiegato Adam come ottimizzatore e la L1Loss (Equation \ref{eq:l1loss}) come loss function.
%Abbiamo scelto di impostare il parametro batch size a 10, learning rate $\lambda$ a 0.01, 100
%come numero massimo di epoche e 20 come patience per l'Early Stopping.

\begin{gather}\label{eq:l1loss}
	L = \{l_1,\dots,l_N\}^\top, \quad
	l_n = \left| x_n - y_n \right|,\\
	\ell(x,y) = \operatorname{mean}(L)
\end{gather}

\begin{table}[H]
	\begin{center}
		\begin{tabular}[c]{l|l}

			\textbf{Total Parameters (\#)}     & 26543400 \\
			\textbf{Trainable Parameters (\#)} & 26543400 \\
			\textbf{Training Duration (s)}     & 24.0     \\
			\textbf{Model Size (MB)}           & 101.3
		\end{tabular}
	\end{center}
	\caption{Baseline Model specification.}\label{tab:ufcnspecs}
\end{table}

\begin{figure}[H]
	\centering
	\includegraphics[width=\textwidth]{chapters/3_models/imgs/ufnc/ufnctraining.png}
	\caption{The chart displays the loss progression during the training phase. The blue line represents the Training Loss, while the orange line represents the Validation Loss.}
	\label{fig:ufcntraining}
\end{figure}

\begin{figure}[H]
	\centering
	\begin{subfigure}{0.32\textwidth}
		\centering
		\includegraphics[width=\textwidth]{chapters/3_models/imgs/ufnc/ufcncpusage.png}
	\end{subfigure}
	\begin{subfigure}{0.32\textwidth}
		\centering
		\includegraphics[width=\textwidth]{chapters/3_models/imgs/ufnc/ufcnmem.png}
	\end{subfigure}
	\begin{subfigure}{0.32\textwidth}
		\centering
		\includegraphics[width=\textwidth]{chapters/3_models/imgs/ufnc/ufcnusagevera.png}
	\end{subfigure}\\
	\begin{subfigure}{0.32\textwidth}
		\centering
		\includegraphics[width=\textwidth]{chapters/3_models/imgs/ufnc/ufncusageperc.png}
	\end{subfigure}
	\begin{subfigure}{0.32\textwidth}
		\centering
		\includegraphics[width=\textwidth]{chapters/3_models/imgs/ufnc/ufncusagew.png}
	\end{subfigure}
	\begin{subfigure}{0.32\textwidth}
		\centering
		\includegraphics[width=\textwidth]{chapters/3_models/imgs/ufnc/ufcnmemram.png}
	\end{subfigure}\\
	\caption{System resources utilized during the Training phase.}
	\label{fig:ufcnsysusage}
\end{figure}

\begin{algorithm}[H]
	\caption{MLP model Training Algorithm}\label{alg:ufcntraining}
	\begin{algorithmic}
		\Require train/validation datasets; Baseline Neural Network Model

		\State Batch Size $\gets$ 10
		\State Learning Rate $\lambda \gets$ 0.01
		\State Epochs $\gets$ 100
		\State Patience $\gets$ 20
		\State loss $\gets$ L1Loss()
		\State Optimizer $\gets$ Adam Optimizer
		\State
		\For{\textbf{each} epoch \textbf{in} epochs}
		\For{\textbf{each} (batch\_id, before, after, target) \textbf{in} train.next\_batch()}

		\State train\_prediction $\gets$ model(before, after) \Comment{Model inference}
		\State train\_prediction $\gets \frac{\text{train\_prediction} \cdot \sum\text{train\_prediction}}{\sum target}$ \Comment{Area normalization}
		\State train\_loss $\gets$ loss(train\_prediction, target)
		\State Optimizer step
		\State Back Propagation
		\EndFor
		\State stop computing gradient
		\For{\textbf{each} (batch\_id, vbefore, vafter, vtarget) \textbf{in} validation.next\_batch()}
		\State val\_prediction $\gets$ model(vbefore, vafter) \Comment{Model inference}
		\State val\_prediction $\gets \frac{\text{val\_prediction} \cdot \sum\text{val\_prediction}}{\sum vtarget}$ \Comment{Area normalization}
		\State val\_loss $\gets$ loss(val\_prediction, vtarget)
		\EndFor

		\State check for Early Stopping
		\State check for Save Best Result
		\State start computing gradient
		\EndFor
	\end{algorithmic}
\end{algorithm}

\subsection{Evaluation}
From the training phase graph shown in Figure~\ref{fig:ufcntraining},
we can see that it has concluded successfully because the two curves
(training loss and validation loss) do not diverge but remain at a
relatively constant distance from each other.
This also suggests that there is no overfitting.
However, the values of the two loss functions do not appear to decrease
significantly as the epochs progress, indicating that the model may
not have generalized well.

%Dal grafico della fase di training mostrato in Figura~\ref{fig:ufcntraining}
%possiamo evicnere come questa sia terminata con successo dal
%fatto che le due curve (training loss e validation loss) non divergono
%ma rimangono ad una distanza abbastanza costante e possiamo
%anche capire che non ci sia presenza di overfitting.
%I valori delle due loss però non sembrano scendere col passare
%delle epoche ma rimangono pressochè costanti, indice che il modello
%potrebbe non essere riuscito a generalizzare bene.

From the graphs in Figure~\ref{fig:ufcnsysusage}, it is apparent that the
machine on which this model was trained was relatively underutilized.
In particular, it's worth noting that GPU utilization did not exceed 30\%,
and GPU memory was barely utilized, staying at around 20\%.
This suggests that the model can be trained on less powerful machines as well.
The model is also relatively lightweight in terms of memory,
with just 100 MB (see Table~\ref{tab:ufcnspecs}), and the training time
did not exceed 30 seconds.
Inference time is extremely fast, taking less than a second.

%Dai grafici in Figura~\ref{fig:ufcnsysusage} possiamo notare come
%la macchina su cui è stato addestrato questo modello sia stata
%sfruttata relativamente poco, in particolare facciamo notare come
%l'utilizzo della GPU non ha superato il 30\% e la sua memoria è stata
%a malapena occupata per il 20\%. Questo può portarci ad affermare che
%il modello può essere anche addestrato su macchine molto meno performanti
%della nostra. Il modello risulta essere anche relativamente leggero sia
%in termini di memoria, soli 100 MB (vedi Tabella~\ref{tab:ufcnspecs}) sia
%per il tempo di allenamento che non ha superato i 30 secondi. Il tempo di
%inferenza risulta estremamente veloce non superando il secondo.

\begin{figure}[H]
	\centering
	\begin{subfigure}{\textwidth}
		\centering
		\includegraphics[width=\textwidth]{chapters/3_models/imgs/ufnc/eval/ufcpred2-4.png}
		\caption{}
	\end{subfigure}
	\begin{subfigure}{\textwidth}
		\centering
		\includegraphics[width=\textwidth]{chapters/3_models/imgs/ufnc/eval/ufcpred6-4.png}
		\caption{}
	\end{subfigure}
	\caption{The figure displays two predictions made by the model, where you can observe the network's output (in red) and the ground truth (in blue). The model was tested on two gaps, each covering a period of 2 days. The first gap (a) spans from 02-04-2023 to 04-04-2023, and the second gap (b) spans from 06-04-2023 to 06-08-2023.}

	%La figura riporta due predizioni del modello dove possiamo apprezzare l'output della rete (in rosso) e la ground trought (in blu). Al modello sono stati fatti chiudere due buchi di 2 giorni ciuascuno che coprono rispettivamente il periodo dal 02-04-2023 al 04-04-2023 e dal 06-04-2023 al 06-08-2023.}
	\label{fig:ufcnevalbelli}
\end{figure}

Analyzing the graphs presented in Figure~\ref{fig:ufcnevalbelli},
we can confirm the previous suspicion that the model did not learn well how
to predict the trend of instant energy production from the plant.
Plot (a) clearly illustrates how the model's prediction is very timid
and fails to grasp the possibility that some days may be more productive
than others.
In graph (b), we can see how the model almost approximates the second
day but makes a significant mistake on the first,
failing to understand the potential for production spikes during the day.
It's also evident that the model struggles to handle the
day/night cycle, often ending production much earlier than it should.
However, it consistently predicts zero energy production
during the night, which is a positive aspect.

%Analizzando i grafici mostrati in Figura~\ref{fig:ufcnevalbelli} possiamo
%confermare il sospetto precedente, il modello non è riuscito ad apprendere
%bene come predirre l'andamento della curva dell'energia istantanea prodotta
%dall'impianto. Il plot (a) mostra molto bene come la predizione del modello sia
%molto timida e non riesce minimamente a comprendere la possibilità che alcuni
%giorni possono essere più produttivi di altri. Mentre nel grafico (b) possiamo
%vedere come il modello riesce quasi ad approssimare il secondo giorno, ma sbagliando
%notevolmente nel primo non risucendo a comprendere la possibilità di picchi di produzione durante la giornata.
%Si evince inoltre come questo ha difficoltà nel gestire il ciclo giorno/notte
%dato che molto spesso fa terminare la produzione molto prima di quando dovrebbe.
%Ottimo però il fatto che la produzione notturna risulti sempre nulla.

\begin{figure}[H]
	\centering
	\begin{subfigure}{\textwidth}
		\centering
		\includegraphics[width=\textwidth]{chapters/3_models/imgs/ufnc/eval/ufcpred3-4.png}
		\caption{}
	\end{subfigure}\\
	\begin{subfigure}{\textwidth}
		\centering
		\includegraphics[width=\textwidth]{chapters/3_models/imgs/ufnc/eval/ufcpred5-4.png}
		\caption{}
	\end{subfigure}
	\caption{The figure displays two model predictions (in red) for two-day gaps and their respective ground truth (in blue). (a) corresponds to a gap from 03-04-2023 to 04-04-2023, while (b) covers the period from 05-04-2023 to 06-04-2023.}
	%In figura sono mostrati due predizioni del modello (in rosso) relative a buchi di due giorni e le rispettive ground truth (in blu). (a) fa riferimento ad un buco che va dal 03-04-2023 fino al 04-04-2023, mentre (b) copre il periodo che va dal 05-04-2023 fino al 06-04-2023.}
	\label{fig:ufcnevalbrutti}
\end{figure}

Figure~\ref{fig:ufcnevalbrutti} highlights the significant limitations
of the model, demonstrating how it fails to identify potential peak
values in production and how the predicted day curves are consistently similar
to each other.
It's also important to note that the model can only work with
gaps of a fixed size, and if a gap of a larger size is encountered,
it would be necessary to retrain the model.

%La Figura~\ref{fig:ufcnevalbrutti} mette in risalto i notevoli limiti del modello
%mostrando come non riesca minimamente ad individuare possibili valori di picco della 
%produzione e come le curve dei giorni predetti siano sempre molto simili tra di loro.
%\'{E} anche opportuno segnalare il fatto che il modello possa solo lavorare con
%buchi di dimensione sempre fissa e, qualora si presentasse un buco di più gradi 
%dimensioni sia necessario dover riallenare nuovamente il modello.

\section{RNN based model Evaluation}\label{sec:rnneval}
From the graphs shown in Figure~\ref{fig:grruntraining},
we can see that the training phase was successful,
and after an initial descent, the loss values remained relatively
constant without displaying any abnormal trends.
The statistics presented in Figure~\ref{fig:grrunsysusage} reveal
that the machine at our disposal was not fully utilized,
suggesting that this architecture can be trained and used on
less powerful computers than the one we have.
Additionally, the model appears to be very lightweight, both in terms
of the relatively small number of parameters and its weight,
which doesn't exceed 400~KB. The inference time does not even
approach one second, making it extremely fast in execution.

\begin{table}[H]
	\begin{center}
		\begin{tabular}[c]{l|l|l}
			%\cline{2-4}
			\multicolumn{1}{c|}{\textbf{Gap Period}} &
			\multicolumn{1}{c|}{\textbf{MAE (kW)}}   &
			\multicolumn{1}{c}{\textbf{R}$^2$}                     \\
			\hline

			02-04 to 05-04                           & 4.72 & 0.96 \\
			04-04 to 04-04                           & 3.60 & 0.93 \\
			12-04 to 14-04                           & 6.61 & 0.96
			% 06-04 to 07-04 & 20.07 & 0.62 &1293.53&50.70&1293.53&25.22 \\
		\end{tabular}
	\end{center}
	\caption{The table displays the values of MAE (Mean Absolute Error) and the R$^2$ (R-squared) index applied to the model predictions shown in Figures~\ref{fig:grrunevalplots}.}\label{tab:grrunpmaer}
	%La tabella mostra i valori del MAE e dell'indice R$^2$ applicate alle predizioni del modello mostrate nelle Figure~\ref{fig:ufcnevalbelli} e \ref{fig:ufcnevalbrutti}.}\label{tab:dfsplit}
\end{table}

\begin{table}[H]
	\begin{minipage}[t]{.45\textwidth}
		\begin{center}
			\begin{tabular}[c]{l|l|l}
				\multicolumn{3}{c}{\textbf{\textit{02-04 to 05-04 Gap}}}       \\
				%\hline
				               & \multicolumn{1}{c|}{\textbf{MAE (kW)}} &
				\multicolumn{1}{c}{\textbf{MAPE (\%)}}                         \\
				\hline
				\textbf{Day 1} & 97.03                                  & 1.99 \\
				\textbf{Day 2} & 28.68                                  & 3.22 \\
				\textbf{Day 3} & 69.79                                  & 4.79 \\
				\textbf{Day 4} & 41.65                                  & 1.98
			\end{tabular}
			%\caption{}
		\end{center}
	\end{minipage}%
	\hfill
	\begin{minipage}[t]{.45\textwidth}
		\begin{center}
			\begin{tabular}[c]{l|l|l}
				\multicolumn{3}{c}{\textbf{\textit{12-04 to 14-04 Gap}}}       \\
				%\hline
				               & \multicolumn{1}{c|}{\textbf{MAE (kW)}} &
				\multicolumn{1}{c}{\textbf{MAPE (\%)}}                         \\
				\hline
				\textbf{Day 1} & 151.93                                 & 2.86 \\
				\textbf{Day 2} & 49.44                                  & 1.01 \\
				\textbf{Day 3} & 111.25                                 & 2.76
			\end{tabular}
			%\caption{}
		\end{center}
	\end{minipage}%
	\vspace{.5cm}
	\begin{minipage}{\textwidth}
		\begin{center}
			\begin{tabular}[c]{l|l|l}
				\multicolumn{3}{c}{\textbf{\textit{04-04 to 04-04 Gap}}}       \\
				%\hline
				               & \multicolumn{1}{c|}{\textbf{MAE (kW)}} &
				\multicolumn{1}{c}{\textbf{MAPE (\%)}}                         \\
				\hline
				\textbf{Day 1} & 0.93                                   & 0.06
			\end{tabular}
			%\caption{}
		\end{center}
	\end{minipage}
	\caption{The tables display the Daily MAE\cite{metrics} and MAPE\cite{metrics} values related to the graphs shown in Figure~\ref{fig:grrunevalplots}.}\label{tab:grrundailymetrics}
	%Nelle tabelle sono mostrati i valori giornalieri di MAE e MAPE relativi ai grafici mostrati in Figura~\ref{fig:grrunevalplots}.}
\end{table}

%Dai grafici mostrati in Figura~\ref{fig:grruntraining} possiamo vedere come la fase di addestramento
%è andata a buon fine e, dopo una parte iniziale di discesa, i valori delle loss sono rimasti pressochè costanti e non mostrano andamenti anomali. Dalle statistiche riportate in Figura~\ref{fig:grrunsysusage} possiamo vedere come la macchina a nostra disposizione non è
%stata sfruttata a pieno e questo può portarci a pensare che questa architettura sia allenabile ed
%utilizzabile su computer meno perfromanti di quello a nostra disposizione. Inoltre il modello 
%risulta essere molto leggero, sia per il numero relativamente ridotto di parametri sia per il 
%peso che non supera i 400 KB.

\begin{figure}[H]
	\centering
	\begin{subfigure}{\textwidth}
		\centering
		\includegraphics[width=\textwidth]{chapters/3_models/imgs/grrun/eval/grruneval24.png}
		\caption{}
	\end{subfigure}
	\begin{subfigure}{\textwidth}
		\centering
		\includegraphics[width=\textwidth]{chapters/3_models/imgs/grrun/eval/grruneval44.png}
		\caption{}
	\end{subfigure}
	\begin{subfigure}{\textwidth}
		\centering
		\includegraphics[width=\textwidth]{chapters/3_models/imgs/grrun/eval/grruneval124.png}
		\caption{}
	\end{subfigure}
	\caption{In the figure, three model predictions (in red) are shown alongside the ground truth (in blue) for gaps of varying sizes. These predictions were made using data from the testing dataset.}
	%In figura vengono mostrate 3 predizioni del modello (in rosso) comparate con la ground thorught (in blu) di buchi a dimensione variabile. Queste predizioni sono state effettuate con i dati provenienti dal dataset di Testing.}
	\label{fig:grrunevalplots}
\end{figure}
\newpage
Analyzing some model predictions as shown in
Figure~\ref{fig:grrunevalplots}, we can observe how the real
instant energy production curves (displayed in blue) are
closely approximated by the model (red curves).
The model effectively understands the plant's behavior,
even managing to predict production spikes.
We can also see that energy production is consistently zero during
the night in the predictions, and it adeptly captures the
day/night cycle by gradually reducing production as sunset approaches.

In graph (a), we can see a 4-day gap from 02-04-2023 to 06-04-2023.
The first two days are predicted very well, while in the last part,
we can see that a production spike was not detected.
In graph (b), we can observe a 1-day gap on 04-04-2023. We notice that the overall trend is almost entirely approximated correctly, except for some time intervals around 12:00.
The last graph (c) is related to a 3-day gap, and we can see that the first two days are approximated well, while in the last day, the production spikes are identified but with values not entirely similar to those of the ground truth.
%Analizzando alcune predizioni del modello riportate in Figura~\ref{fig:grrunevalplots} possiamo notare come le curve dell'energia 
%istantanea prodotta realmente (mostrate in blu) vengono approssimate decisamente
%bene dal modello (curve in rosso). Il modello riesce a capire bene il comportamento
%dell'impianto riuscendo anche a prevedere eventuali picchi di produzione.
%Possiamo notare come di notte la produzione di energia nelle predizioni è sempre nulla e come riesce molto bene a comprendere il ciclo giorno/notte
%andando ad azzerare gradualmente la produzione quando si avvicina il tramonto.

%Nel grafico (a) possiamo vedere un buco di 4 giorni che va dal 02-04-2023 al 06-04-2023. I primi due giorni vengono predetti molto bene, mentre nell'ultimo possiamo vedere che un picco di produzione non è stato individuato.
%Nel prolt (b) possiamo apprezzare un buco di 1 giorno, il 04-04-2023.
%Notiamo come l'andamento è quasi del tutto approsiamoto correttamente tranne
%che per alcuni intervalli temporali che si aggirano intorno le 12:00.
%L'ultimo grafico (c) è relativo ad un buco di 3 giorni e possiamo vedere
%come i primi due vengono approssimati bene, mentre nell'ultimo i picchi
%di produzione vengono si individuati ma con valori non del tutto simili
%a quelli della ground truth.

\begin{figure}[H]
	\centering
	\begin{subfigure}{\textwidth}
		\centering
		\includegraphics[width=.9\textwidth]{chapters/3_models/imgs/grrun/eval/grruneval6buco.png}
		\caption{}
	\end{subfigure}
	\begin{subfigure}{\textwidth}
		\centering
		\includegraphics[width=.9\textwidth]{chapters/3_models/imgs/grrun/eval/grruneval12buco.png}
		\caption{}
	\end{subfigure}
	\caption{The graphs depict two model predictions for gaps that exceed the maximum limit of days set during the training phase. The first one (a) shows a 6-day gap, while the second one (b) presents a 12-day gap.}
	%I grafici mostrano due predizioni del modello di buchi con dimensioni che superano il limite massimo di giorni impostati nella fase di training. Il primo (a) mostra un buco di 6 giorni, mentre il secondo (b) uno di 12.}
	\label{fig:grrunevalbucogrande}
\end{figure}

It is interesting to note how the model still performs well even
when presented with gaps that exceed the maximum length set
during training.
In Figure~\ref{fig:grrunevalbucogrande}, two graphs are shown:
(a) represents a 6-day gap (two days longer than the maximum length),
and (b) a 12-day gap.
Given these results, we can conclude that the model can
generalize effectively and predict instant energy production
trends with considerable reliability.


%\'{E} interessante notare come il modello riesce a perforare comunque bene anche se gli vengono passati dei buchi che superano la dimensione massima impostata durante l'addestramento. In Figura~\ref{fig:grrunevalbucogrande} vengono mostrati due grafici, (a) è un buco di 6 giorni (due in più della dimensione massima), mentre (b) è di 12. Dati questi risultati possiamo
%affermare che il modello è in grado di generalizzare molto bene e riuscire
%a predirre l'andamento dell'energia istantanea prodotta con notevole affidabilità.

\begin{table}[H]
	\begin{minipage}[t]{.45\textwidth}
		\begin{center}
			\begin{tabular}[t]{l|l|l}
				\multicolumn{3}{c}{\textbf{\textit{03-04 to 08-04 Gap}}} \\
				%\hline
				               &
				\makecell{\textbf{MAE}                                   \\\textbf{(kW)}} &
				\makecell{\textbf{MAPE}                                  \\\textbf{(\%)}} \\
				%\multicolumn{1}{c|}{\textbf{MAE (kW)}} &
				%\multicolumn{1}{c}{\textbf{MAPE (\%)}} \\
				\hline
				\textbf{Day 1} & 56.76  & 6.37                           \\
				\textbf{Day 2} & 82.02  & 5.63                           \\
				\textbf{Day 3} & 59.90  & 2.84                           \\
				\textbf{Day 4} & 150.79 & 5.91                           \\
				\textbf{Day 5} & 12.52  & 0.24                           \\
				\textbf{Day 6} & 238.88 & 6.73
			\end{tabular}
			%\caption{}
		\end{center}
	\end{minipage}%
	\hfill
	\begin{minipage}[t]{.45\textwidth}
		\begin{center}
			\begin{tabular}[t]{l|l|l}
				\multicolumn{3}{c}{\textbf{\textit{02-04 to 13-04 Gap}}} \\
				%\hline
				                &
				\makecell{\textbf{MAE}                                   \\\textbf{(kW)}} &
				\makecell{\textbf{MAPE}                                  \\\textbf{(\%)}} \\
				%\multicolumn{1}{c|}{\textbf{MAE (kW)}} &
				%\multicolumn{1}{c}{\textbf{MAPE (\%)}} \\
				\hline
				\textbf{Day 1}  & 148.45 & 3.04                          \\
				\textbf{Day 2}  & 38.14  & 4.28                          \\
				\textbf{Day 3}  & 55.41  & 3.80                          \\
				\textbf{Day 4}  & 20.26  & 0.96                          \\
				\textbf{Day 5}  & 104.28 & 4.09                          \\
				\textbf{Day 6}  & 112.09 & 2.18                          \\
				\textbf{Day 7}  & 312.26 & 8.79                          \\
				\textbf{Day 8}  & 47.65  & 3.46                          \\
				\textbf{Day 9}  & 16.32  & 0.99                          \\
				\textbf{Day 10} & 23.08  & 0.43                          \\
				\textbf{Day 11} & 240.66 & 4.52                          \\
				\textbf{Day 12} & 22.21  & 0.45
			\end{tabular}
			%\caption{}
		\end{center}
	\end{minipage}
	\caption{The tables presented refer to the graphs in Figure~\ref{fig:grrunevalbucogrande} and display the corresponding Daily MAE and MAPE metrics.}\label{tab:grrunbuchigrandi}
	%Le tabelle mostrate fanno riferimento ai grafici in Figura~\ref{fig:grrunevalbucogrande} e mostrano le relative metriche Daily MAE e MAPE.}
	%Nelle tabelle sono mostrati i valori giornalieri di MAE e MAPE relativi ai grafici mostrati in Figura~\ref{fig:grrunevalplots}.}
\end{table}

Taking into account the data in Tables~\ref{tab:grrunpmaer}
and \ref{tab:grrundailymetrics}, it can be confirmed that this
Recurrent Neural Network-based model is highly performant.
The Gap MAE consistently remains low, not exceeding 10 kW,
while the $R^2$ index almost always approaches the ideal value of 1.0\cite{metrics}, never dropping below 0.70 (a value reached only in rare cases). Furthermore, the Daily MAPE values barely exceed 6\%, indicating very low daily error. The Global MAPE is 3.17\%, and the Global MAE is 95.75 kW, demonstrating excellent overall performance.

%Prendendo in esame le Tabelle~\ref{tab:grrunpmaer} e \ref{tab:grrundailymetrics} possiamo confermare che questo modello basato
%su Recurrent Neaural Network sia molto performante. Vediamo come
%il \textit{Gap MAE} riamnga sempre basso, non superando i 10 kW, mentre 
%l'indice $R^2$ si avvicini quasi sempre al valore ideale di 1.0\cite{metrics} e comunque non scendendo mai sotto a 0.70 (valore raggiunto solo in rarissimi casi).
%Inoltre i valori del Daily MAPE superano a malapena il 6\% indicando un
%bassissimo errore giornaliero che globalmente risulta essere di 3.17\% per 
%il \textit{Global MAPE} e di 95.75 kW per il \textit{Global~MAE}.

Equally positive results are obtained for gaps that exceed the
maximum length of 4 days.
From Table~\ref{tab:grrunbuchigrandi}, we can see that for the
12-day gap, the Daily MAPE never exceeds 9\%, occasionally even reaching as low as 0.45\% error. The largest error is found on day 7, with a value of 312.26 kW.
Similarly, for the 6-day gap, the MAPE values do not exceed 10\%,
and the maximum error is found on the last day, with around
239 kW (6.73\%).
For both graphs, we have an $R^2$ value greater than 0.90, specifically 0.92 and 0.97, with a Gap MAE of 6.56 and 6.37 kW.

In conclusion, it can be affirmed that this model effectively
harnesses the potential offered by Recurrent Neural Networks and
excels in predicting the trend of instant energy production
during data gaps, even for gaps spanning multiple days.
%Altrettanto positivi sono i dati dei buchi che eccedono la dimensione massima di 4 giorni. Dalla Tabella~\ref{tab:grrunbuchigrandi}
%possiamo vedere che per il buco di 12 giorni il Daily MAPE non supera mai
%il 9\% raggungendo alle volte anche il 0.45\% di errore. L'errore più grande
%è relativo al giorno 7 per un valore di 312.26 kW.
%Similmente per il buco da 6 giorni abbiamo valori del MAPE che non superano
%il 10\% e l'errore massimo lo troviamo nell'ultimo giorno con circa 239 kW (6.73\%). Per tutti e due i grafici abbiamo un valore dell'indice $R^2$ superiore
%al .90, relativamente 0.92 e 0.97 con un Gap MAE di 6.56 e 6.37 kW.

%In conclusione possiamo affermare che questo modello riesce a sfruttare a pieno
%le potenzialità offerte dalle Reti Ricorrenti e riesce notevolmente bene
%a predirre l'andamento dell'energia istantanea prodotta durante bunchi di
%dati anche di molteplici giorni.
\section{Transformer based model Evaluation}
From the graph shown in Figure~\ref{fig:gabtrainchart}, we can observe that the training and validation loss curves, after an initial descent, remain relatively constant, and most importantly, they do not seem to diverge from each other. This indicates that the training phase has concluded successfully, with the validation loss value being lower by 0.01 compared to the RNN-based model.

From the graphs in Figure~\ref{fig:gabsysusage}, which show the machine's usage during the training phase, and from Table~\ref{tab:gabspecs}, it can be inferred that the model utilizes almost all of the available hardware resources, suggesting that it is computationally intensive and might be challenging to train on less powerful machines. However, it's essential to note that the inference time is extremely fast, not exceeding one second, and the model file size is very compact, with a dimension of approximately 2 MB.
%Dal grafico mostrato in Figura~\ref{fig:gabtrainchart} possiamo vedere
%come le curve della training e validation loss dopo un'iniziale discesa
%rimangono pressocche costanti e soprattutto non sembrano divergere tra
%di loro. Questo ci mostra come la fase di training si sia conclusa con successo con anche il valore della validation loss inferiore di 0.01 rispotto a
%quello del modello basato su RNN.

%Dai grafici in Figura~\ref{fig:gabsysusage} che mostrano l'utilizzo della macchina durante la fase 
%di addestramento e dalla Tabella~\ref{tab:gabspecs} si evince che
%il modello sfrutta quasi del tutto l'hardware a sua dispisizione 
%suggerendoci che questo risulta essere abbastanza oneroso in termini
%di risorse computazionali e che difficilimente possa essere addestrato
%con macchine meno performanti. E' però importante notare che il
%tempo di inferenza risulta
%estremamente veloce non superando il secondo di tempo e il peso 
%del file del modello è molto contenuto con una dimensione di circa 2 MB.

\begin{table}[H]
	\begin{center}
		\begin{tabular}[c]{l|l|l|l}
			%\cline{2-4}
			\multicolumn{1}{c|}{\textbf{Gap Period}} &
			\multicolumn{1}{c|}{\textbf{MAE (kW)}}   &
			\multicolumn{1}{c|}{\textbf{MAPE (\%)}}  &                     % * 100
			\multicolumn{1}{c}{\textbf{R}$^2$}                             \\
			\hline

			05-04 10:15 to 07-04 17:45               & 5.02 & 22.38 & 0.98 \\
			01-04 10:00 to 01-04 19:45               & 7.24 & 11.72 & 0.96 \\
			02-04 09:30 to 04-04 08:15               & 3.14 & 25.11 & 0.99 \\
			05-04 03:00 to 06-04 01:30               & 4.02 & 24.40 & 0.96 \\
			% 06-04 to 07-04 & 20.07 & 0.62 &1293.53&50.70&1293.53&25.22 \\
		\end{tabular}
	\end{center}
	\caption{The table displays the values of MAE (Mean Absolute Error), MAPE (Mean Absolute Percentage Error) and the R$^2$ (R-squared) index applied to some model predictions made during testing phase shown in Figures~\ref{fig:gabsolobuchi}.}\label{tab:gabpmaer}
	%La tabella mostra i valori del MAE e dell'indice R$^2$ applicate alle predizioni del modello mostrate nelle Figure~\ref{fig:ufcnevalbelli} e \ref{fig:ufcnevalbrutti}.}\label{tab:dfsplit}
\end{table}

\begin{table}[H]
	\begin{minipage}{.5\textwidth}
		\centering
		\begin{tabular}{l|c}
			\multicolumn{2}{c}{\textit{variable gap size}}            \\
			                       & \multicolumn{1}{c}{\textbf{AVG}} \\
			\hline
			\textbf{AVG MAE (kW)}  & 3.83 $\pm$ 1.40                  \\
			\textbf{AVG MAPE (\%)} & 18.49 $\pm$ 7.99                 \\
			\textbf{AVG R$^2$}     & 0.97 $\pm$ 0.05
		\end{tabular}
		\caption*{(a)}
	\end{minipage}%
	\begin{minipage}{.5\textwidth}
		\centering
		\begin{tabular}{l|c}
			\multicolumn{2}{c}{\textit{2 days gap size}}              \\
			                       & \multicolumn{1}{c}{\textbf{AVG}} \\
			\hline
			\textbf{AVG MAE (kW)}  & 3.76 $\pm$ 0.39                  \\
			\textbf{AVG MAPE (\%)} & 18.14 $\pm$ 6.76                 \\
			\textbf{AVG R$^2$}     & 0.98 $\pm$ 0.02
		\end{tabular}
		\caption*{(b)}
	\end{minipage}%
	\caption{Table (a) displays the global metrics calculated on the model's output with variable gap sizes, while (b) refers to the global metrics with a fixed two-day gap size. Both tables include the mean value and the standard deviation.}
	% La Tabella (a) mostra le metriche globali calcolate sull'output del modello con gap size di dimensione variabile, mentre (b) fa riferimento alle metriche globali con gap size fissa a due giorni. In entrabe le tabelle oltre al valor medio viene anche riportata la deviazione standard.}
	\label{tab:gabglobalmetrics}
\end{table}

\begin{figure}[H]
	\centering
	\begin{subfigure}{\textwidth}
		\centering
		\includegraphics[width=\textwidth]{chapters/3_models/imgs/gab/eval/gabplotall1.png}
		\caption{}
	\end{subfigure}
	\begin{subfigure}{\textwidth}
		\centering
		\includegraphics[width=\textwidth]{chapters/3_models/imgs/gab/eval/gabplotall2.png}
		\caption{}
	\end{subfigure}
	\begin{subfigure}{\textwidth}
		\centering
		\includegraphics[width=\textwidth]{chapters/3_models/imgs/gab/eval/gabplotall3.png}
		\caption{}
	\end{subfigure}
	\caption{This figure shows some of the input time series provided to the model during the testing phase. We can see the entire series (in blue), the ground truth only for the gap period (in red), and the model's output only in the relevant gap period (in green). In graph (a), we have a gap size of almost two days, in (b) almost one day, and in (c) just over half a day.}
	%In questa Figura vengono mostrate alcune serie temporali passate in input al modello durante la fase di testing. Possiamo vedere l'intera serie (in blu), la ground truth solo del buco (in rosso) e l'output del modello solo nel periodo di buco interessato (in verde). Nel grafico (a) abbiamo una gap size di quasi due giorni, in (b) di quasi un giorno ed in (c) di poco più di metà giorno. }
	\label{fig:gaballplotsgaps}
\end{figure}


\begin{figure}[H]
	\centering
	\begin{subfigure}{\textwidth}
		\centering
		\includegraphics[width=.95\textwidth]{chapters/3_models/imgs/gab/eval/gabbuco5.png}
		\caption{}
	\end{subfigure}
	\begin{subfigure}{\textwidth}
		\centering
		\includegraphics[width=.95\textwidth]{chapters/3_models/imgs/gab/eval/gabbuco2.png}
		\caption{}
	\end{subfigure}
	\begin{subfigure}{\textwidth}
		\centering
		\includegraphics[width=.95\textwidth]{chapters/3_models/imgs/gab/eval/gabbuco3.png}
		\caption{}
	\end{subfigure}
	\begin{subfigure}{\textwidth}
		\centering
		\includegraphics[width=.95\textwidth]{chapters/3_models/imgs/gab/eval/gabbuco4.png}
		\caption{}
	\end{subfigure}
	\caption{This figure displays some results of the model obtained during the testing phase, using gaps of varying lengths. You can see the model's outputs (in red) compared to their respective ground truths (in blue).}
	%Nella figura sono mostrati alcuni risultati del modello ottenuti durante la fase di testing, utilizzando gap size di lunghezza variabile. Possiamo vedere gli output del modello (in rosso) confrontati con le relative ground truth (in blu).}
	\label{fig:gabsolobuchi}
\end{figure}

Analyzing the results of the evaluation phase presented partly in Table~\ref{tab:gabpmaer} and in the graphs in Figure~\ref{fig:gabsolobuchi}, we can see that this model is very effective in predicting instantaneous energy during gaps of varying sizes. It excels in this task, whether for gaps of nearly a day or up to three days. We can observe that the $R^2$ index in these cases is consistently high, almost always hovering around the value of 0.97, with relatively low MAE and MAPE values.

Examining the global metrics reported in Table~\ref{tab:gabglobalmetrics} (a), we can confirm these findings. It shows an extremely low average MAE of 3.83 kW with a standard deviation of 1.40 kW, an average MAPE of 18.49\% with a standard deviation of 7.99\%, and an average $R^2$ value of 0.97 $\pm$ 0.05. This demonstrates how the model effectively approximates the ground truth with very low errors.

%Analizzando i risultati della fase di valutazione mostrati in 
%parte nella Tabella~\ref{tab:gabpmaer} e dai grafici in Figura~\ref{fig:gabsolobuchi} possiamo vedere come questo
%modello sia molto performante nel predirre l'energia istantanea durante
%buchi di differente dimensione. Si nota come riesce bene in
%questo compito sia su buchi di quasi un giorno fino ad arrivare
%anche a tre giorni. Possiamo vedere come l'indice $R^2$ in questi casi sia notevolmente
%alto, aggirandosi quasi sempre sul valore di 0.97 con anche
%valori di MAE e MAPE relativamente bassi.
%Analizzando le metriche globali riportate nella Tabella~\ref{tab:gabglobalmetrics} (a) possiamo confermare queste affermazioni.
%Presenta un MAE medio estremamente contenuto di 3.83 kW con una 
%deviazione standard di 1.40 kW, un MAPE medio di 18.49\% con deviazione standard di 7.99\% ed un valore medio di 0.97$\pm 0.05$ per l'indice $R^2$.

%unrisulta estremamente
%variabile con una valor medio di 16.67\% e una deviazione standar di 25.26\%. Lo stesso si applica per l'indice $R^2$
%che presenta si un buon valore di 0.92 ma una deviazione
%standard decisamente elevata di 0.23.
%Queste performance non brillanti sono dovute dal fatto che
%il modello faccia fatica a individuare l'andamento di buhi
%con dimensioni estremamente ridotte come mostrato in Figura~\ref{} con le relative metriche riportate in Tabela~\ref{}.
%
%E' importante notare che il modello presenta alcune difficoltà
%nell'individuare il corretto valore del primo e dell'ultimo valore
%del buco. Difficoltà che non si presentano estremente di frequente
%e che non comportano gravi ripercussioni sulle performance di questo.
%
%L'architettura presenta però problemi nel gestire buchi di
%piccole dimensioni, non superiori ai 90 timestamp (che rappresentano quasi un giorno intero). In questi casi possiamo anche raggiungere
%valori di 0.60 per l'indice $R^2$ ed anche un errore medio del 40\%.
%
%\begin{figure}[H]
%    \centering
%    \includegraphics[width=\textwidth]{chapters/3_models/imgs/gab/eval/gabbucobrutto.png}
%    \caption{Nella figura viene mostrato un esempio di buco di piccole dimensioni dove il modello riscontra problemi nella predizione dell'energia istantanea. In questo caso abbiamo un MAE di 1.26 kW, un valore di 0.61 per l'indice $R^2$ e un MAPE del 45\%.}
%    \label{fig:enter-label}
%\end{figure}

\begin{figure}[H]
	\centering
	\begin{subfigure}{\textwidth}
		\centering
		\includegraphics[width=\textwidth]{chapters/3_models/imgs/gab/eval/2days/gab2gionri54.png}
		\caption{}
	\end{subfigure}
	% \begin{subfigure}{\textwidth}
	%     \centering    
	%     \includegraphics[width=.7\textwidth]{chapters/3_models/imgs/gab/eval/2days/gab2giorni64.png}
	%     \caption{}
	% \end{subfigure}
	\begin{subfigure}{\textwidth}
		\centering
		\includegraphics[width=\textwidth]{chapters/3_models/imgs/gab/eval/2days/gab2giorni104.png}
		\caption{}
	\end{subfigure}
	\caption{In this figure, some outputs from the testing phase of the model with a fixed 2-day gap size are presented. The model's predictions are shown in red, while the ground truth is in blue.}
	%Nella figura sono riportati alcuni output della fasi di testing del modello con gap size fissa a 2 giorni. Le predizioni del modello sono mostrate in rosso mentre la ground turth in blu.}
	%\caption{This figure displays some results of the model obtained during the testing phase, using gaps of varying lengths. You can see the model's outputs (in red) compared to their respective ground truths (in blue).}
	%Nella figura sono mostrati alcuni risultati del modello ottenuti durante la fase di testing, utilizzando gap size di lunghezza variabile. Possiamo vedere gli output del modello (in rosso) confrontati con le relative ground truth (in blu).}
	\label{fig:gabsolobuchi2giorni}
\end{figure}
\newpage
By repeating the evaluation phase with fixed gap sizes of two days to allow for a comparison with the other models, we can notice a slight improvement in the model's performance. Analyzing the data presented in Table~\ref{tab:gabglobalmetrics} (b), we have an average MAE of 3.76 $\pm$ 0.39 kW (2\% improvement with a 72\% better standard deviation), an average MAPE of 18.14 $\pm$ 6.76\% (2\% improvement with a 15\% better standard deviation), and an average $R^2$ value of 0.98 $\pm$ 0.02 (1\% improvement). In general, there is a 2\% overall increase in performance with a significant reduction in the standard deviation values.

%Ripetendo nuovamente la fase di valutazione e fissando a due giorni 
%le dimensioni dei buchi generabili, per permettere poi un confronto
%con gli altri modelli, possiamo notare un leggero incremento nell performance
%del modello. Analizzando i dai riportati in Tabella~\ref{tab:gabglobalmetrics} (b) abbiamo un
%MAE medio di 3.76$\pm 0.39$ kW (migliore del 2\% con deviazione standard migliore del 72\%), un MAPE medio di 18.14$\pm 6.76$ \% (migliore del 2\% con deviazione standard
%migliore del 15\%) ed un valor medio per l'indice $R^2$ di
%0.98$\pm 0.02$ (migliore del dell'1\%). In generale possiamo
%riscontrare un aumento delle performance generale del 2\%
%con un notevole abbassamento del valore delle deviazioni 
%standard.

\begin{table}[H]
	\begin{center}
		\begin{tabular}[c]{l|l|l|l}
			%\cline{2-4}
			\multicolumn{1}{c|}{\textbf{Gap Period}} &
			\multicolumn{1}{c|}{\textbf{MAE (kW)}}   &
			\multicolumn{1}{c|}{\textbf{MAPE (\%)}}  &                     % * 100
			\multicolumn{1}{c}{\textbf{R}$^2$}                             \\
			\hline

			04-04 to 05-04                           & 3.72 & 30.48 & 0.96 \\
			05-04 to 06-04                           & 4.25 & 26.99 & 0.96 \\
			06-04 to 07-04                           & 4.59 & 20.39 & 0.98 \\
			07-04 to 08-04                           & 4.33 & 15.29 & 0.98 \\
			08-04 to 09-04                           & 3.72 & 22.86 & 0.97 \\
			09-04 to 10-04                           & 3.52 & 29.76 & 0.94 \\
			10-04 to 11-04                           & 3.88 & 20.91 & 0.99

			% 06-04 to 07-04 & 20.07 & 0.62 &1293.53&50.70&1293.53&25.22 \\
		\end{tabular}
	\end{center}
	\caption{The table displays the values of MAE (Mean Absolute Error), MAPE (Mean Absolute Percentage Error), and the R$^2$ (R-squared) index applied to some model predictions made during the testing phase with a fixed gap size of 2 days. Some of these predictions are shown in Figures~\ref{fig:gabsolobuchi2giorni}.}
	%\label{tab:gabpmaer}
	%La tabella mostra i valori del MAE e dell'indice R$^2$ applicate alle predizioni del modello mostrate nelle Figure~\ref{fig:ufcnevalbelli} e \ref{fig:ufcnevalbrutti}.}\label{tab:dfsplit}
\end{table}

In conclusion, we can affirm that, despite the more demanding training phase, this model excels in the task of estimating instant energy production during periods of significantly variable gap sizes, achieving significantly better MAE, MAPE, and R$^2$ index values compared to those of the previous architectures.

%In conclusione possiamo affermare che, nonostante la più impegnativa fase di training, questo modello eccelle
%nel compito di stimare l'energia istantanea prodotta durante periodi
%di buchi con dimensione notevolmente variabile, riscontrando valori di MAE, MAPE e indice $R^2$ nettamente migliori rispetto a quelli delle precedenti architetture.
%

%\begin{table}[H]
%    \centering
%    \begin{tabular}{l|l|l|l|l}
%    \multicolumn{1}{c|}{\textbf{Gap}} &
%    \multicolumn{2}{c|}{\textbf{SAE (kW)}} &
%    \multicolumn{2}{c}{\textbf{SAPE (\%)}} \\
%    \hline
%    & Day 1 & Day 2 & Day 1 & Day 2\\
%    \hline
%
%    03-04 to 04-04 & 44.40 & 44.40 & 4.98 & 03.04\\
%    04-04 to 05-04 & 02.51 & 02.51 & 0.17 & 00.11\\
%    05-04 to 06-04 & 79.61 & 79.61 & 3.78 & 03.12\\
%    06-04 to 07-04 & 39.49 & 39.49 & 1.54 & 00.76\\
%    07-04 to 08-04 & 71.99 & 71.99 & 1.40 & 02.02\\
%    08-04 to 09-04 & 139.91 & 139.91 & 3.94 & 10.16\\
%    09-04 to 10-04 & 37.49 & 37.49 & 2.72 & 02.27
%    \end{tabular}
%    \caption{Nella tabella vengono riportati i risultati della fase di valutazione con gap size fissa a due giorni.}
%    \label{tab:gabdailyerrori}
%\end{table}

% DEV STANDARD
%\newpage
\section{MLP vs RNN}
We will now compare the experimental results obtained by the
MLP and RNN-based models.
To do this, we will retest both networks on the testing dataset,
forcing the generation of gaps that are consistently two days
long to facilitate comparison (see Section~\ref{sec:mlpbaseline}
and Section~\ref{sec:rnnbasemodel}).
We will compare their performance by calculating, for each of them,
the pointwise error between ground truth and prediction.
Additionally, we will compare the various other metrics used
and discuss some strengths and weaknesses of each model.

%Confronteremo ora i risultati sperimentali ottenuti
%dai modelli basati su MLP e RNN. Per farlo andremo a testare nuovamente
%le due reti sul dataset di testing forzando la generazione di buchi
%sempre a 2 giorni, in modo tale da poter confrontare i risultati (vedi 
%Sezione~\ref{sec:mlpbaseline} e \ref{sec:rnnbasemodel}). Metteremo quindi
%a confronto le performance di questi andando a calcolare, per ognuno,
%l'errore puntuale tra ground truth e predizione. 
%Infine, oltre a confrontare tra di loro le altre varie metriche impiegate,
%andremo a discutere alcuni punti di forza e debolezze di ogni modello.

\begin{table}[H]
	\centering
	\begin{tabular}{l|l|l|l}
		\multicolumn{1}{c}{}                         &
		\multicolumn{2}{c}{\textit{lower is better}} &                                                   \\
		\multicolumn{1}{c|}{\textbf{Gap}}            &
		\multicolumn{2}{c|}{\textbf{MAE (kW)}}       &
		\multicolumn{1}{c}{\textbf{Gain (\%)}}                                                           \\ %% ((MLP - RNN)*100)/MLP
		\hline
		                                             & \small \textbf{MLP} & \small \textbf{RNN} &       \\
		\cline{2-3}
		02-04 to 03-04                               & 18.63               & 3.91                & 79.01 \\
		03-04 to 04-04                               & 06.90               & 3.29                & 52.31 \\
		04-04 to 05-04                               & 10.41               & 5.55                & 46.68 \\
		05-04 to 06-06                               & 17.99               & 6.89                & 61.70 \\
		06-04 to 07-04                               & 20.07               & 6.78                & 66.21 \\
		07-04 to 08-04                               & 14.77               & 9.25                & 37.37 \\
		08-04 to 09-04                               & 15.82               & 7.85                & 50.39 \\
		09-04 to 10-04                               & 10.30               & 5.82                & 43.49
	\end{tabular}
	\caption{This table highlights the difference between the error of the MLP-based model and the RNN-based model. The column \texttt{Gain (\%)} represents the absolute percentage difference between MLP MAE and RNN MAE.}
	%In questa tabella viene messo in evidenza la differenza tra l'errore del modello basato su MLP e quello su Reti Ricorrenti. La colonna \texttt{Diff (kW)} è la differenza in valore assoluto tra MLP MAE e RNN MAE, mentre \texttt{Diff (\%)} è la differenza in percentuale.}
	\label{tab:mlpvsrnndiff}
\end{table}

\begin{table}[H]
	\begin{minipage}[t]{.45\textwidth}
		\centering
		\begin{tabular}[t]{l|l|l|l}
			\multicolumn{1}{c}{}                         &
			\multicolumn{2}{c}{\textit{lower is better}} &                                                  \\
			\multicolumn{1}{c|}{\textbf{Gap}}            &
			\multicolumn{2}{c|}{\makecell{\textbf{Max Err.}                                                 \\\textbf{(kW)}}} &
			\makecell{\textbf{Gain}                                                                         \\\textbf{(\%)}} \\
			\hline
			                                             & \small \textbf{MLP} & \small\textbf{RNN} &       \\
			\cline{2-3}
			02-04 to 03-04                               & 90.74               & 35.74              & 60.61 \\ %55.0 \\
			03-04 to 04-04                               & 96.29               & 21.69              & 77.47 \\ %74.60\\
			04-04 to 05-04                               & 89.01               & 54.95              & 29.27 \\ %26.06\\
			05-04 to 06-06                               & 105.01              & 53.91              & 48.66 \\ %51.10\\
			06-04 to 07-04                               & 122.56              & 51.35              & 58.10 \\%71.21\\
			07-04 to 08-04                               & 105.94              & 75.91              & 28.34 \\ % 30.03\\
			08-04 to 09-04                               & 86.14               & 75.47              & 06.55 \\%05.65\\
			09-04 to 10-04                               & 78.76               & 59.97              & 23.85 %18.79
		\end{tabular}
		\caption*{(a)}
		%\caption{Nella tabella viene mostrata la differenza tra i valori dell'indice $R^2$ per i due modelli.}
	\end{minipage}%
	\hfill
	\begin{minipage}[t]{.45\textwidth}
		\centering
		\begin{tabular}[t]{l|l|l|l}
			\multicolumn{1}{c}{}                         &
			\multicolumn{2}{c}{\textit{lower is better}} &                                                  \\
			\multicolumn{1}{c|}{\textbf{Gap}}            &
			\multicolumn{2}{c|}{\makecell{\textbf{Min Err.}                                                 \\\textbf{(kW)}}} &
			\makecell{\textbf{Gain}                                                                         \\\textbf{(\%)}} \\
			\hline
			                                             & \small \textbf{MLP} & \small\textbf{RNN} &       \\
			\cline{2-3}
			02-04 to 03-04                               & 7.58                & 0.90               & 88.12 \\ %6.68 \\
			03-04 to 04-04                               & 0.79                & 0.05               & 93.67 \\ %0.74 \\
			04-04 to 05-04                               & 0.39                & 0.85               & 50.11 \\ %-0.46\\
			05-04 to 06-06                               & 1.45                & 0.05               & 96.55 \\ %1.40 \\
			06-04 to 07-04                               & 4.84                & 0.13               & 97.31 \\ %4.71 \\
			07-04 to 08-04                               & 1.73                & 1.39               & 19.65 \\ %0.34 \\
			08-04 to 09-04                               & 5.19                & 0.89               & 82.85 \\% 4.30\\
			09-04 to 10-04                               & 3.19                & 0.71               & 77.74 \\ %2.48
		\end{tabular}
		\caption*{(b)}
	\end{minipage}
	\begin{minipage}{\textwidth}
		\vspace{.1cm}
		\centering
		\begin{tabular}[t]{l|l|l|l}
			\multicolumn{1}{c}{}                          &
			\multicolumn{2}{c}{\textit{bigger is better}} &                                                  \\
			\multicolumn{1}{c|}{\textbf{Gap}}             &
			\multicolumn{2}{c|}{\textbf{$R^2$}}           &
			\multicolumn{1}{c}{\textbf{Gain (\%)}}                                                           \\
			\hline
			                                              & \small \textbf{MLP} & \small\textbf{RNN} &       \\
			\cline{2-3}
			02-04 to 03-04                                & 0.64                & 0.98               & 34.69 \\ %-0.34\\
			03-04 to 04-04                                & 0.52                & 0.91               & 42.85 \\ %-0.39\\
			04-04 to 05-04                                & 0.52                & 0.89               & 41.57 \\%-0.37\\
			05-04 to 06-06                                & 0.42                & 0.91               & 53.84 \\ %-0.49\\
			06-04 to 07-04                                & 0.62                & 0.95               & 37.73 \\ %-0.33\\
			07-04 to 08-04                                & 0.78                & 0.91               & 14.28 \\ %-0.13\\
			08-04 to 09-04                                & 0.55                & 0.85               & 35.29 \\ %-0.30\\
			09-04 to 10-04                                & 0.42                & 0.78               & 46.15 %-0.36
		\end{tabular}
		\caption*{(c)}
	\end{minipage}

	\caption{The tables compare the results of the two models, highlighting the maximum pointwise error (a), the minimum pointwise error (b), and the $R^2$ index (c). For each metric, the absolute percentage difference is also shown.}
	%Le tabelle mettono a confronto i risultati dei due modelli paragonando il massimo errore puntuale (a) espresso in kW, il minimo errore puntuale (b) espresso in kW e l'indice $R^2$ (c). Per ognuna viene mostrata anche la differenza tra i due valore ($MPL - RNN$).}
\end{table}

From the previous tables, it's evident that the RNN-based model achieves significantly higher $R^2$ values compared to the MLP-based model. On average, the RNN-based model's values hover around 0.90, while the MLP-based model is around 0.60, with an average difference of 30\%. The second model approximates the trend of instant energy production during gaps much better.

In tables (a) and (b), it can be observed that the architecture utilizing RNN has maximum and minimum error values significantly lower than the MLP-based model. Regarding Gap MAE, the second model is also superior to the first by approximately 60\%, with significantly lower average errors.

%Dalle precedenti tabelle possiamo vedere com il modello basato su RNN ottiene
%predizioni con un indice $R^2$ molto superiore a quello del modello basato su MLP. In media i valori del primo si aggirano intorno a 0.90 mentre per 
%il secondo siamo sul 0.60 con una differenza media di 0.30 punti. Il secondo modello approssima nettamente meglio l'andamento dell'energia istantanea prodotta durante i buchi. 
%Dalle tabelle (a) e (b) si vede come l'architettura che sfrutta RNN 
%ha valori di errore massimi e minimi decisamente più contenuti rispetto a quella che
%si basa su MLP.
%Anche per il Gap MAE possiamo vedere come il secodo modello
%sia migliore rispetto al primo di circa il 60\% commettendo in media errori
%nettamente più bassi rispetto all'altro.

\begin{figure}[H]
	\centering
	\begin{subfigure}{\textwidth}
		\centering
		\includegraphics[width=\textwidth]{chapters/4_evaluation/imgs/mlpvsrnn/mlpvsrnn1.png}
		\caption{}
	\end{subfigure}
	\begin{subfigure}{\textwidth}
		\centering
		\includegraphics[width=\textwidth]{chapters/4_evaluation/imgs/mlpvsrnn/mlpvsrnn2.png}
		\caption{}
	\end{subfigure}
	\begin{subfigure}{\textwidth}
		\centering
		\includegraphics[width=\textwidth]{chapters/4_evaluation/imgs/mlpvsrnn/mlpvsrnn3.png}
		\caption{}
	\end{subfigure}
	\caption{The figure displays the difference curves between the target and prediction for the MLP-based models (in blue) and the RNN-based models (in red).}
	% Nella figura sono mostrate le curve delle differenze tra target e predizione dei modelli basati su MLP (mostrate in blu) e su RNN (in rosso).}
	\label{fig:mlpvsrnngrafici}
\end{figure}
From the graphs shown in Figure~\ref{fig:mlpvsrnngrafici},
it's immediately noticeable that the architecture utilizing
recurrent neural networks commits significantly lower errors
than the MLP-based model.
The red curves are generally more compact and closer to 0
(no error committed), while the blue curves are on average
much farther from the ideal value, indicating a significantly
higher pointwise error, on average by 60\%, with the presence
of substantial peaks (b).
It's important to highlight that both models do not commit errors at night.

Comparing the respective outputs of the two models,
it's evident that the predicted curves by the MLP-based model
are extremely similar to each other and fail to identify
possible production peaks.
On the other hand, the output of the second model
performs significantly better in the task, generating curves
that closely follow the ground truth, understanding the
presence of potential energy peaks, and demonstrating the
ability to generate curves with different shapes and widely
varying values.

It's also noticeable that the first model struggles with
identifying day and night periods, often ending production a
few hours before sunset.
In contrast, the second model excels in understanding
this alternation and does not exhibit issues of this nature.

It's also important to note that both models are relatively
lightweight in terms of resources for training and inference
on the architecture at our disposal.
This suggests the possibility of training and using
them on less powerful machines.

The second model can also close gaps of considerably larger
sizes than those used during the training phase.
In contrast, the first model can only handle 2-day gaps,
and to increase this limit, the model would need to repeat
the training phase.

%Dai grafici mostrati in Figura~\ref{fig:mlpvsrnngrafici} possiamo 
%notare immediatamente che l'architettura che sfrutta le reti ricorrenti
%commette errori significativamente più bassi di quella basata su MLP.
%Notiamo come le curve in rosso sono generalmente più compatte ed abbastanza
%vicine allo 0 (nessun errore commesso). Mentre quelle in blu risultano mediamente
%molto più distanti dal valore ideale evidenziando quindi un errore puntuale nettamente più alto, in media del 60\%, con la presenza di picchi di notevole valore (b). \'{E} importante evidenziare che di notte tutti e
%due i modelli non commettono errori.

%Confrontando i relativi output dei due modelli possiamo notare come
%le curve predette dal modello basato su MLP siano estremamente simili
%tra di loro e non riescono ad individuare possibili picchi di produzione.
%Mentre l'output del secondo riesce nettamente meglio nel compito generando
%curve che seguno bene l'andamento della ground trouth, riuscendo a comprendere
%anche presenza di possibili picchi di energia prodotta oltre all'abilità
%di generare curve di differente forma e valori anche moltro diversi
%tra di loro.

%Notiamo anche che il primo modello ha qualche difficoltà nel capire 
%i periodi di giorno e notte con il risultato che molto spesso termina la 
%produzione qualche ora prima del tramonto, mentre il secondo riesce molto
%bene a comprendere questa alternanza e non presenta probemi di questo tipo.

%\'{E} importante anche notare che tutti e due i modelli risultano relativamente
%leggeri sia in termini di risorse per l'allenamento e per l'inferenza sull'architettura a nostra disposizione. Suggerendo quindi la possibilità
%di essere addestrati ed usati anche su macchine meno performanti.

%Il secondo modello riesce anche a chiudere buchi di dimensioni notevolmente
%maggiori rispetto a quelli utilizzati durante la fase di training, mentre il primo riesce a gestire solo buchi di 2 giorni e per poter aumentare questo limite, il modello necessita di ripetere nuovamente la fase di trianing.

\begin{figure}[H]
	\centering
	\begin{subfigure}{\textwidth}
		\centering
		\includegraphics[width=.75\textwidth]{chapters/3_models/imgs/ufnc/eval/ufcpred5-4.png}
		\caption{}
	\end{subfigure}
	\begin{subfigure}{\textwidth}
		\centering
		\includegraphics[width=.75\textwidth]{chapters/3_models/imgs/grrun/eval/grruneval124.png}
		\caption{}
	\end{subfigure}
	\caption{Visual comparison between an output of the MLP-based model (a) and an output of the RNN-based model (b).}
	%Confronto visuale tra un output del modello basato su MLP (a) e output del modello basato su RNN (b).}
\end{figure}

%\section{Final Conclusions}


In conclusion, the models described and evaluated in this work have shown varying degrees of success in predicting the instant energy production curve of the plant during gaps. The Multi-Layer Neural Network-based model, despite being computational lightweight and efficient, demonstrated limited ability to generalize and predict production variations effectively. It struggled to capture the possibility of production spikes and had difficulty managing the day/night cycle.

While the Recurrent Neural Network-based model exhibited promise, particularly in capturing diverse energy production variations and handling more complex and variable patterns, it outshone the other model in various ways. This model can effectively manage gaps of various sizes, even surpassing the dimensions it was initially trained on, all without necessitating a repeated training phase. Furthermore, it demonstrates the ability to discern the day/night cycle and adapt production output accordingly.

Overall, the models provide valuable insights into energy production trends, and their predictability can be useful in real photovoltaic implants contexts. However, further model refinement and optimization, along with the introduction of techniques that can handle variable gap sizes and more complex patterns, may be necessary to achieve more accurate and reliable predictions. Additionally, expanding the dataset and applying advanced techniques for time series prediction could contribute to more successful models in the future.