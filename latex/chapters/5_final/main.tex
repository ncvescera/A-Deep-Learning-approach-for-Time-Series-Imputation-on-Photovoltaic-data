\chapter{Final Conclusions}
In conclusion, the models described and evaluated in this work have shown varying degrees of success in predicting the instant energy production curve of the plant during gaps. The Multi-Layer Perceptron Neural Network-based model, as expected, demonstrated limited ability to generalize and predict production variations effectively. It struggled to capture the possibility of production spikes and had difficulty managing the day/night cycle.

The Recurrent Neural Network-based model exhibited promise, particularly in capturing diverse energy production variations and handling more complex and variable patterns, it outshone the MLP-based model in various ways. This model can effectively handle gaps of various sizes, ranging from just a few timestamps to a large number of days. Furthermore, it demonstrates the ability to distinguish the day/night cycle and adjust production output accordingly. Its training phase is extremely fast and lightweight, making it very suitable for practical applications in addressing this problem, despite slightly higher errors.

Lastly, the absolute best model is the one based on the Transformer architecture. Its low error compared to the others and excellent ability to handle time series of highly variable lengths make it ideal for this task. It excels in approximating the energy produced by the plant during data gaps, comprehending the day/night alternation well and predicting potential energy spikes. Unfortunately, the training phase is quite resource-intensive, but once this hurdle is overcome, the inference time is extremely low.
%Infine, il modello migiore in assoluto risulta essere quello basato su Transformer che, con un basso errore commesso rispetto agli altri e l'ottima capagità nel gestire time series di lunghezza estremamente variabili lo rendono davvero perfetto per risolvere questo compito. Riesce notevolmente bene nell'approssiamre l'energia prodotta dall'impianto durante i buchi di dati, comprendendo molto bene l'alternaza giorno/notte e riuscendo a prevedere la presenza di potenziali picchi di energia. Purtroppo la fase di training risulta abbastanza onerosa a livello hardware ma, una volta superato questo scoglio il tempo di inferenza è estremamente basso.

Overall, the models provide valuable insights into energy production trends, and their predictability can be useful in real photovoltaic implants contexts. However, further model refinement and optimization, along with the introduction of techniques that can handle variable gap sizes and more complex patterns, may be necessary to achieve more accurate and reliable predictions. Additionally, expanding the dataset and applying advanced techniques for time series prediction could contribute to more successful models in the future.