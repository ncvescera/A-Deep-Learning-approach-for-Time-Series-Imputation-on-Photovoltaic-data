\chapter{Dataset and Data Preprocessing}\label{chap:datapreprocessing}

In this chapter, we will analyze the dataset in detail, focusing on its structure, the various features it contains, and the procedures we have adopted to transform it into a new database with a better and more efficient format for use in model training and evaluation. We will also see how the Open-Meteo meteorological service was used and, most importantly, how the most important features were selected to enable the training of various models.

%In questo capitolo analizzeremo nel dettaglio il dataset a nostra disposizione soffermandoci sulla sua struttura, le varie feature contnute al suo interno, e le varie procedure che abbiamo adottato per poterlo trasformare in una nuova base di dati con un formato migliore e più efficiente per essere impiegata nell'addestramento e valutazione dei modelli. Vedremo anche come è stato utilizzato il servizio meteorologico OpenMeteo e soprattutto come sono state poi selezionate le feature più importanti per rendre possibile l'addestramento dei vari modelli.

\section{Dataset}

Il dataset a nostra disposizione descrive un periodo di quasi due anni (dal
02/02/2022 al 16/06/2023) ed è proveniente da un impianto fotovoltaico da 978 kW
situato nella provincia di Bari. È composto da
3 inverter, 27 quadri di campo, 1 contatore, 1 solarimetro, 2 protezioni
interfaccia e 1 dispositivo \textquote{impianto} nel quale vengono immagazzinati
i dati provenienti da Solargis. Il dataset è organizzato in file, uno per
tipologia di dispositivo, che
rappresentano la singola giornata e i dati sono aggregati a 5 minuti ed ogni
riga riporta il riferimento al
dispositivo di appartenenza (\verb|deviceName| e \verb|deviceId|).

\begin{table}[H]
	\begin{center}
		\begin{tabular}[c]{l}
			\hline
			\textbf{File Name}                                  \\
			\hline
			\verb|2022_02_02_Rofilo_NP00003174_inverter.csv|    \\
			\verb|2022_02_02_Rofilo_NP00003174_meteorology.csv| \\
			\verb|2022_02_02_Rofilo_NP00003174_meter.csv|       \\
			\verb|2022_02_02_Rofilo_NP00003174_other.csv|       \\
			\verb|2022_02_02_Rofilo_NP00003174_plantDevice.csv| \\
			\verb|2022_02_02_Rofilo_NP00003174_stringbox.csv|   \\
			\verb|2022_02_03_Rofilo_NP00003174_inverter.csv|    \\
			\verb|2022_02_03_Rofilo_NP00003174_meteorology.csv| \\
			\verb|2022_02_03_Rofilo_NP00003174_meter.csv|       \\
			\verb|2022_02_03_Rofilo_NP00003174_other.csv|       \\
			\verb|2022_02_03_Rofilo_NP00003174_plantDevice.csv| \\
			\verb|2022_02_03_Rofilo_NP00003174_stringbox.csv|   \\
			$\ldots$                                            \\
			\hline
		\end{tabular}
	\end{center}
	\caption{Estratto di alcuni file del dataset.}\label{tab:datasunto}
\end{table}

%% TODO: migliorare dati
\begin{table}[H]
	\begin{center}
		\begin{tabular}[c]{l|l|l|l|l|l}
			\hline
			\multicolumn{1}{c|}{\textbf{timestamp}}        &
			\multicolumn{1}{c|}{\textbf{serial}}           &
			\multicolumn{1}{c|}{\textbf{$\ldots$}}         &
			\multicolumn{1}{c|}{\textbf{TotalEnergy(kWh)}} &
			\multicolumn{1}{c|}{\textbf{Frequency(Hz)}}    &
			\multicolumn{1}{c}{\textbf{deviceid}}            \\
			\hline
			01/02/2023 00:05                               &
			INV01                                          &
			$\ldots$                                       &
			431324.36                                      &
			49.88                                          &
			83204                                            \\

			01/02/2023 00:10                               &
			INV01                                          &
			$\ldots$                                       &
			431324.36                                      &
			49.88                                          &
			83204                                            \\

			01/02/2023 00:15                               &
			INV01                                          &
			$\ldots$                                       &
			431324.36                                      &
			49.88                                          &
			83204                                            \\

			$\ldots$                                       &
			$\ldots$                                       &
			$\ldots$                                       &
			$\ldots$                                       &
			$\ldots$                                       &
			$\ldots$                                         \\
			01/02/2023 12:00                               &
			INV01                                          &
			$\ldots$                                       &
			431324.36                                      &
			49.88                                          &
			83204                                            \\
			01/02/2023 12:05                               &
			INV01                                          &
			$\ldots$                                       &
			431324.36                                      &
			49.88                                          &
			83204                                            \\
			01/02/2023 12:10                               &
			INV01                                          &
			$\ldots$                                       &
			431324.36                                      &
			49.88                                          &
			83204                                            \\

			$\ldots$                                       &
			$\ldots$                                       &
			$\ldots$                                       &
			$\ldots$                                       &
			$\ldots$                                       &
			$\ldots$                                         \\
			01/02/2023 23:45                               &
			INV01                                          &
			$\ldots$                                       &
			431324.36                                      &
			49.88                                          &
			83204                                            \\
			01/02/2023 23:50                               &
			INV01                                          &
			$\ldots$                                       &
			431324.36                                      &
			49.88                                          &
			83204                                            \\
			01/02/2023 23:55                               &
			INV01                                          &
			$\ldots$                                       &
			431324.36                                      &
			49.88                                          &
			83204                                            \\
			\hline
		\end{tabular}
		\caption{Contenuto di un file che rappresenta un inverter}\label{tab:invsunto}
	\end{center}
\end{table}

\subsection{Inverter}
Un inverter in un impianto fotovoltaico è un componente fondamentale che svolge
una funzione vitale: converte l'energia elettrica prodotta dai pannelli solari,
che è in corrente continua (DC), in energia elettrica utilizzabile in corrente
alternata (AC). Questa conversione è cruciale perché la maggior parte delle
apparecchiature elettriche domestiche e delle reti di distribuzione elettrica
utilizzano corrente alternata per funzionare.
Regolano la
tensione e la frequenza della corrente alternata prodotta per garantire che
siano conformi agli standard di rete elettrica locale. Questo è importante
per evitare danni agli apparecchi elettrici e per garantire
l'interoperabilità con la rete elettrica.
Sono spesso dotati di
tecnologie avanzate come il "Maximum Power Point Tracking" (MPPT) che
ottimizzano costantemente la produzione di energia solare. Questo significa
che l'inverter regola la tensione in ingresso dai pannelli solari per
ottenere la massima potenza possibile in base alle condizioni di luce solare
in tempo reale.
Le feature che caratterizzano il nostro impianto sono riassunte dalla seguente
tabella:

\begin{table}[H]
	\begin{center}
		\begin{tabular}[c]{l|l|l}
			\hline
			\multicolumn{1}{c|}{\textbf{Name}}        &
			\multicolumn{1}{c|}{\textbf{Unit Symbol}} &
			\multicolumn{1}{c}{\textbf{Description}}                                         \\
			\hline
			CommunicationCode                         & -   & COMMUNICATION CODE             \\
			Failure 3                                 & -   & Allarme Attivo                 \\
			Failure 4                                 & -   & Allarme di isolamento          \\
			CurrentDC                                 & A   & Corrente di campo fotovoltaico \\
			CurrentAC                                 & A   & Corrente di Rete               \\
			CurrentAC Phase1                          & A   & Corrente RMS di Linea Fase R   \\
			CurrentAC Phase2                          & A   & Corrente RMS di Linea Fase S   \\
			CurrentAC Phase3                          & A   & Corrente RMS di Linea Fase T   \\
			TotalEnergy                               & kWh & Energia Attiva Erogata         \\
			Frequency                                 & Hz  & Frequenza di Rete              \\
			PowerAC Phase1                            & kW  & PA di Linea Fase R             \\
			PowerAC Phase2                            & kW  & PA di Linea Fase S             \\
			PowerAC Phase3                            & kW  & PA di Linea Fase T             \\
			PowerAC                                   & kW  & PA Erogata                     \\
			PowerDC                                   & kW  & Potenza di campo fotovoltaico  \\
			Status                                    & -   & Stato Inverter                 \\
			Failure                                   & -   & Stato PLL per Aggancio Rete    \\
			Failure 2                                 & -   & Stato Rete 1                   \\
			Failure 1                                 & -   & Stato Rete 2                   \\
			InternalTemperature                       & C   & Temp. CPU                      \\
			HeatSinkTemperature                       & C   & Temp. IGBT                     \\
			VoltageDC                                 & V   & Tensione di campo fotovoltaico \\
			VoltageAC                                 & V   & Tensione di Rete               \\
			VoltageAC Phase1                          & V   & Tensione RMS di Linea Fase R   \\
			VoltageAC Phase2                          & V   & Tensione RMS di Linea Fase S   \\
			VoltageAC Phase3                          & V   & Tensione RMS di Linea Fase T   \\
			\hline
		\end{tabular}
		\caption{Lista di tutte le features degli inverter, con descrizione.}\label{tab:invfeatures}
	\end{center}
\end{table}

%%TODO: aggiungere grafico inverter TotalEnergy, Temperature, e altri ?

\subsection{StringBox}
Le stringbox in un impianto fotovoltaico sono contenitori elettrici progettati
per ospitare e proteggere una serie (o stringa) di pannelli solari collegati in
serie. Questi dispositivi svolgono diverse funzioni importanti:

\begin{itemize}
	\item Protezione: Le stringbox includono dispositivi di protezione come interruttori
	      automatici o fusibili che prevengono cortocircuiti e sovraccarichi nel circuito
	      fotovoltaico.
	\item Connettori: Forniscono connettori sicuri per collegare i cavi provenienti dai
	      pannelli solari alla stringa di cavi principale dell'impianto.
	\item Monitoraggio: Alcune stringbox sono dotate di sistemi di monitoraggio che
	      consentono di rilevare prestazioni o guasti dei pannelli solari all'interno
	      della stringa.
	\item Isolamento: Possono anche includere dispositivi di isolamento che consentono di
	      interrompere l'alimentazione elettrica verso la stringa di pannelli solari per
	      scopi di manutenzione o sicurezza.
\end{itemize}

\subsection{JunctionBox}

\begin{table}[H]
	\begin{center}
		\begin{tabular}[c]{l|l|l}
			\hline
			\multicolumn{1}{c|}{\textbf{Name}}        &
			\multicolumn{1}{c|}{\textbf{Unit Symbol}} &
			\multicolumn{1}{c}{\textbf{Description}}                                           \\
			\hline
			CommunicationCode                         & -              & COMMUNICATION CODE    \\
			Failure                                   & -              & Allarme Stringhe      \\
			CurrentString1                            & A              & Corrente I1           \\
			CurrentString2                            & A              & Corrente I2           \\
			CurrentString3                            & A              & Corrente I3           \\
			CurrentString4                            & A              & Corrente I4           \\
			CurrentString5                            & A              & Corrente I5           \\
			CurrentString6                            & A              & Corrente I6           \\
			CurrentString7                            & A              & Corrente I7           \\
			AverageStringCurrent                      & A              & Corrente Media        \\
			Irradiance                                & $\text{W/m}^2$ & Irraggiamento moduli  \\
			Failure 1                                 & -              & Stringhe Aperte       \\
			Failure 2                                 & -              & Stringhe Non Perform. \\
			EnvironmentTemperature                    & C              & Temperatura ambiente  \\
			ModuleTemperature                         & C              & Temperatura moduli    \\
			InternalTemperature                       & C              & Temperatura Scheda    \\
			\hline
		\end{tabular}
		\caption{Lista di tutte le features deglle JunctionBox, con descrizione.}\label{tab:junctionfeatures}
	\end{center}
\end{table}


\subsection{Solargis}
Solargis è una società specializzata nella fornitura di dati e servizi di
previsione solare per impianti fotovoltaici e progetti legati all'energia
solare. Il loro principale obiettivo è fornire informazioni precise e affidabili
sull'irradiazione solare e sulle condizioni meteorologiche solari in qualsiasi
parte del mondo. Questi dati sono fondamentali per la progettazione,
l'ottimizzazione e la gestione degli impianti fotovoltaici.
Solargis raccoglie e fornisce dati dettagliati
sull'irradiazione solare globale, diretta e diffusa in ogni posizione
geografica. Questi dati consentono agli sviluppatori di impianti fotovoltaici di
valutare la quantità di energia solare disponibile in una determinata area, il
che è fondamentale per dimensionare correttamente l'impianto e calcolare le
previsioni di produzione.

\begin{table}[H]
	\begin{center}
		\begin{tabular}[c]{l|l|l|l}
			\hline
			\multicolumn{1}{c|}{\textbf{timestamp}}         &
			\multicolumn{1}{c|}{\textbf{$\ldots$}}          &
			\multicolumn{1}{c|}{\textbf{SolargisGHI(W/m2)}} &
			\multicolumn{1}{c}{\textbf{SolargisGTI(W/m2)}}                          \\
			\hline
			2022-08-01 11:40:00                             & $\ldots$ & 896 & 978  \\
			2022-08-01 11:45:00                             & $\ldots$ & 896 & 978  \\
			2022-08-01 11:50:00                             & $\ldots$ & 896 & 978  \\
			2022-08-01 11:55:00                             & $\ldots$ & 914 & 1001 \\
			2022-08-01 12:00:00                             & $\ldots$ & 914 & 1001 \\
			2022-08-01 12:05:00                             & $\ldots$ & 914 & 1001 \\
			2022-08-01 12:10:00                             & $\ldots$ & 928 & 1019 \\
			2022-08-01 12:15:00                             & $\ldots$ & 928 & 1019 \\
			2022-08-01 12:20:00                             & $\ldots$ & 928 & 1019 \\

			\hline
		\end{tabular}
		\caption{Dati di Solargis a nostra disposizione, provenienti dal file
			\texttt{2022\_08\_01\_Rofilo\_NP00003174\_plantDevice.csv}}\label{tab:solargis}
	\end{center}
\end{table}

%%TODO: aggiungere grafico Solargis

Solar radiation takes a long journey until it reaches Earth’s surface. So when
modelling solar radiation, various interactions of extra-terrestrial solar
radiation with the Earth’s atmosphere, surface and objects are to be taken into
account. The component that is neither reflected nor scattered, and which
directly
reaches the surface, is called direct radiation; this is the component that
produces shadows. The component that is scattered by the atmosphere, and which
reaches the ground is called diffuse radiation. The small part of the radiation
reflected by the surface and reaching an inclined plane is called the reflected
radiation. These three components together create global radiation.

In solar energy applications, the following parameters are commonly used in
practice:

\begin{itemize}
	\item Direct Normal Irradiation/Irradiance (DNI) is the component that is
	      involved in thermal (concentrating solar power, CSP) and photovoltaic
	      concentration
	      technology (concentrated photovoltaic, CPV).
	\item  Global Horizontal
	      Irradiation/Irradiance (GHI) is the sum of direct and diffuse radiation
	      received on a horizontal plane. GHI is a reference radiation for the
	      comparison of climatic zones; it is also essential parameter for
	      calculation of radiation on a tilted plane.
	\item Global Tilted Irradiation/Irradiance (GTI), or total
	      radiation received on a surface with defined tilt and azimuth, fixed or
	      sun-tracking. This is the sum of the scattered radiation, direct and
	      reflected. It is a reference for photovoltaic (PV) applications, and
	      can be occasionally affected by shadow.
\end{itemize}


\begin{table}[H]
	\begin{center}
		\begin{tabular}[c]{l|l|l}
			\hline
			\multicolumn{1}{c|}{\textbf{Name}}        &
			\multicolumn{1}{c|}{\textbf{Unit Symbol}} &
			\multicolumn{1}{c}{\textbf{Description}}                                                            \\
			\hline
			SolargisGHI                               & $\text{W/m}^2$ & Solargis Global Horizontal Irradiation \\
			SolargisGTI                               & $\text{W/m}^2$ & Solargis Global Tilted Irradiation     \\
			\hline
		\end{tabular}
		\caption{Lista di tutte le features di Solargis, con descrizione.}\label{tab:solargisfeatures}
	\end{center}
\end{table}


%Avere il dataset suddiviso in diversi file, uno per giorno e per
dispositivo (vedi Sezione \ref{sec:dataset}), con la presneza di
alcuni periodi, che vanno da pochi minuti a diversi gionri, di assenza dati
(dovuti probabilmente ad un fallimento dello strumento di raccolta dati)
fa sì che si verifichino molti problemi durante la fase di training
e la rende quasi impossibile. In questo capitolo vedremo come abbiamo
risolto questi problemi, optando per una struttura tabellare monolitica
del dataset e quali approccia abbiamo utilizzato per la gestione
dei dati mancanti.
\section{Dataset Realization}
For the creation of our dataset, we devised a procedure that
allowed us to obtain a single file, in CSV format, where for
each timestamp, we have all the data for the entire system at
that exact moment. Below is the final structure of the dataset
and the algorithm for generating it.

%Per la realizzazione del nostro dataset abbiamo ideato 
%una procedura che ci ha permesso di ottenere un unico file,
%in formato csv, dove, per ogni time stamp abbiamo tutti i dati
%di tutto l'impianto in quell'esatto istante. Di seguito 
%la struttira finale del dataset e l'algoritmo per generarlo.


\begin{table}[H]
	\begin{center}
		\begin{tabular}[c]{l|l|l|l}
			\hline
			\multicolumn{1}{c|}{\textbf{timestamp}}              &
			\multicolumn{1}{c|}{\textbf{DEV.NAME$_1$\_FEAT$_1$}} &
			\multicolumn{1}{c|}{$\ldots$}                        &
			\multicolumn{1}{c}{\textbf{DEV.NAME$_n$\_FEAT$_n$}}                                   \\
			\hline

			01/10/2022 10:00                                     & $\ldots$ & $\ldots$ & $\ldots$ \\
			01/10/2022 10:05                                     & $\ldots$ & $\ldots$ & $\ldots$ \\
			01/10/2022 10:10                                     & $\ldots$ & $\ldots$ & $\ldots$ \\
			\hline
		\end{tabular}
	\end{center}
	\caption{Final dataset feature structure.}\label{tab:datasetform}
\end{table}

\begin{algorithm}[H]
	\caption{Dataset aggragation algorithm}\label{alg:dataset}
	\begin{algorithmic}
		\Require data\_folder
		\Ensure \texttt{data\_folder} exists
		\State dev\_types $\gets$ find all file types inside \texttt{data\_folder} (e.g. meter, inverter, $\ldots$)
		\State dev\_content $\gets$ a dictionary with \texttt{dev\_types} types as $keys$ and empty $values$

		\For {\textbf{each} key \textbf{in} dev\_conent.keys}
		\State files $\gets$ find all file matching type \texttt{key} inside \texttt{data\_folder}
		\State sort \texttt{files} by date (asc.)
		\State temp\_type\_aggregate $\gets$ and empty csv table
		\For {\textbf{each} file \textbf{in} files}
		\State append all \texttt{file} lines to \texttt{temp\_type\_aggregate} table
		\EndFor
		\State dev\_content[key] $\gets$ temp\_type\_aggregate
		\EndFor
		\State\Comment{At this time we have a dictionary mapping a file type with all its available data}
		\State
		\State dataset $\gets$ an empty csv table
		\For {\textbf{each} type, data \textbf{in} dev\_content} \Comment{\texttt{type} is $key$, \texttt{data} is $value$}
		\State rename all \texttt{data} $columns$ to \texttt{data.deviceID}\_\texttt{$column$.name}
		\State except for 'timestamp' $column$
		\State dataset $\gets$ merge \texttt{dataset} and \texttt{data} tables using 'timestamp' column
		\EndFor
		\State save \texttt{dataset} table to file
	\end{algorithmic}
\end{algorithm}

\begin{table}[H]
	\begin{center}
		\begin{tabular}[c]{l|l|l|l}
			\hline
			\multicolumn{1}{c|}{\textbf{timestamp}}      &
			\multicolumn{1}{c|}{\textbf{INV01\_PowerAC}} &
			\multicolumn{1}{c|}{\textbf{$\ldots$}}       &
			\multicolumn{1}{c}{\textbf{Cont\_TotalEnergy}}                                 \\
			\hline
			2022-02-02 00:05:00                          & NaN      & $\ldots$ & NaN       \\
			2022-02-02 00:10:00                          & NaN      & $\ldots$ & NaN       \\
			$\ldots$                                     & $\ldots$ & $\ldots$ & $\ldots$  \\
			2022-06-22 10:20:00                          & 175.66   & $\ldots$ & 8900941.5 \\
			2022-06-22 10:25:00                          & 178.29   & $\ldots$ & 8900995.5 \\
			2022-06-22 10:30:00                          & 180.82   & $\ldots$ & 8901036.0 \\
			$\ldots$                                     & $\ldots$ & $\ldots$ & $\ldots$  \\
			2023-06-16 18:00:00                          & NaN      & $\ldots$ & NaN       \\
			2023-06-16 18:05:00                          & NaN      & $\ldots$ & NaN       \\
		\end{tabular}
	\end{center}
	\caption{Some data from dataset after running Algorithm \ref{alg:dataset}}\label{tab:datasetfinalvalues}
\end{table}

\subsection{Timestamp cyclical encoding}
To enable the models to learn the alternation of minutes, days,
and months as effectively as possible during the training phase,
we applied a procedure to transform each individual timestamp into
a pair of sine and cosine values, thus performing a cyclic
encoding of various seasonalities.

%Per permettere ai modelli, durante la fase di allenamento, di apprendere
%al meglio possibile l'alternarsi dei minuti, giorni e mesi abbiamo
%applicato una procedura per trasfrormare ogni signolo timestamp
%in una coppia di valori seno-coseno effettuando così un 
%encoding ciclico delle varie stagionalità.


\begin{algorithm}[H]
	\caption{Cyclical Encoding Algorithm}\label{alg:cyclicencoding}
	\begin{algorithmic}
		\Require dataset table
		\Ensure \texttt{dataset} \textbf{is not} empty
		\State dataset['minute\_sin'] $\gets \sin(2 \pi (\frac{\text{\texttt{dataset.timestamp.minute}}}{60}))$
		\State dataset['minute\_cos'] $\gets \cos(2 \pi (\frac{\text{\texttt{dataset.timestamp.minute}}}{60}))$


		\State dataset['hour\_sin'] $\gets \sin(2 \pi (\frac{\text{\texttt{dataset.timestamp.hour}}}{24}))$
		\State dataset['hour\_cos'] $\gets \cos(2 \pi (\frac{\text{\texttt{dataset.timestamp.hour}}}{24}))$


		\State dataset['day\_sin'] $\gets \sin(2 \pi (\frac{\text{\texttt{dataset.timestamp.day}}}{31}))$
		\State dataset['day\_cos'] $\gets \cos(2 \pi (\frac{\text{\texttt{dataset.timestamp.day}}}{31}))$

		\State dataset['month\_sin'] $\gets \sin(2 \pi (\frac{\text{\texttt{dataset.timestamp.month}}}{12}))$
		\State dataset['month\_cos'] $\gets \cos(2 \pi (\frac{\text{\texttt{dataset.timestamp.month}}}{12}))$
	\end{algorithmic}
\end{algorithm}

\begin{figure}[H]
	\centering
	\includegraphics[width=0.7\linewidth, keepaspectratio]{chapters/2_data_preprocessing/imgs/hoursincosplot.png}
	\caption{Hour cyclical encoding plot.}
	\label{fig:encodingplot}
\end{figure}

\subsection{Dealing with Holes}

\begin{figure}[H]
	\centering
	\begin{subfigure}[t]{0.48\textwidth}
		\centering
		\includegraphics[width=\textwidth, keepaspectratio]{chapters/2_data_preprocessing/imgs/totenergybuco1.png}
	\end{subfigure}
	\hspace{0.1cm}
	\begin{subfigure}[t]{0.48\textwidth}
		\centering
		\includegraphics[width=\textwidth, keepaspectratio]{chapters/2_data_preprocessing/imgs/totenergybuco2.png}
	\end{subfigure}\\

	\begin{subfigure}[t]{0.48\textwidth}
		\centering
		\includegraphics[width=\textwidth, keepaspectratio]{chapters/2_data_preprocessing/imgs/inv02powerbuco1.png}
	\end{subfigure}
	\hspace{0.1cm}
	\begin{subfigure}[t]{0.48\textwidth}
		\centering
		\includegraphics[width=\textwidth, keepaspectratio]{chapters/2_data_preprocessing/imgs/inv02powerbuco1.png}
	\end{subfigure}
	\caption{Some dataset \textquote{holes}. The charts at the top refer to the Implant's Total Energy, while those at the bottom refer to the Inverter 2's Power. The charts on the left range from 01-06-2022 to 15-06-2022, while those on the right range from 12-02-2023 to 19-02-2023.}
	\label{fig:graficibuchi}
\end{figure}

As we can see from Figure \ref{fig:graficibuchi}, there are certain
periods within the dataset (highlighted in red) where data is missing,
which we refer to as \textquote{holes}. Leaving these gaps in the
dataset causes problems during model training
(hindering the correct calculation of the loss function), and therefore,
they need to be removed. One possible approach for holes removal
is to perform a \textit{fill} operation: filling the gap with the
first available non-null value. This tactic may be considered
acceptable only if the time span involves just a few timestamps.
However, if we are talking about several hours or even days, it
significantly distorts the overall production and instantaneous
power trends, resulting, especially in very unfortunate cases, with extremely abnormal production curves.

The solution we have adopted to address this problem is the removal of
the entire day when a hole occurs. For example, if we have a data
gap from 12-02-2023 23:00 to 13-02-2023 10:40, the days
12-02-2023 and 13-02-2023 will be completely removed. With this
method, we lose some data, but as we will see later, the number of
gaps is not extremely high, and this data loss is not
debilitating. The following algorithm summarizes what has
just been described.

%Come possiamo vedere dalla Figura \ref{fig:graficibuchi}, all'interno del dataset sono presenti alcuni periodi (evidenziati in rosso) di assenza dati che è ciò che chiamiamo \textquote{buco}.
%Lasciare questi buchi causa problemi durante l'allenamento dei modelli (impediscono il corretto calcolo della loss function) e per questo
%vanno rimossi. Un possibile approccio per la loro rimozione è effettuare
%un'operazione di \textquote{fill}: riempio il buco con il primo valore non nullo disponibile.
%Questa tattica può essere ritenuta accettabile solo se il lasso di tempo coinvolge solo qualche timestamp, se invece parliamo di
%qualche ora o addirittura giorni questo va ad alterare notevolmente
%l'andamento della produzione totale e della potenza istantanea, risultando in casi molto sfortunati, ad avere curve di produzione estremamente anomale.
%
%La soluzione che abbiamo adottato per risolvere questo problema è
%l'eliminazione di tutto il giorno in cui si presenta il buco.
%Per fare un esempio pratico, se abbiamo un buco di dati che va dal
%12-02-2023 23:00 al 13-02-2023 10:40, verranno completamente eliminati
%i giorni 12-02-2023 e 13-02-2023. Con questo metodo andiamo a perdere 
%alcuni dati, ma come vedremo poi, il numero dei buchi non è estremamente
%elevato e la perdita di questi dati non risulta invalidante. Il seguente algoritmo riassume quanto appena detto.
%
\noindent\begin{minipage}[t]{0.55\linewidth}
	\begin{algorithm}[H]
		\caption{Holes Removal Algorithm.}\label{alg:holes}
		\begin{algorithmic}
			\Require dataset table
			\Ensure \texttt{dataset} \textbf{is not} empty

			\State holes $\gets$ find all timestamp from \texttt{dataset} table, where there are some \texttt{Nan}s inside the columns
			\For{\textbf{each} row \textbf{in} \texttt{dataset.rows}}
			\If {row.timestamp \textbf{is in} \texttt{holes}}
			\State drop \texttt{row} from \texttt{dataset} table
			\EndIf
			\EndFor
		\end{algorithmic}
	\end{algorithm}

\end{minipage}%
\hfill%
\begin{minipage}[t]{0.30\linewidth}
	\begin{table}[H]
		\centering
		\begin{tabular}[c]{l}
			\multicolumn{1}{c}{\textbf{Timestamp}} \\
			\hline
			2022-06-09                             \\
			2022-06-10                             \\
			2022-06-11                             \\
			2022-06-12                             \\
			2022-06-13                             \\
			2022-06-28                             \\
			2022-06-29                             \\
			2022-06-30                             \\
			2022-08-26                             \\
			2022-09-23                             \\
			2022-10-06                             \\
			2023-02-03                             \\
			2023-02-15                             \\
			2023-02-16                             \\
			2023-03-26                             \\
		\end{tabular}
		\caption{Timestamps deleted after running the Algorithm \ref{alg:holes}}
	\end{table}
\end{minipage}


\subsection{Historical weather}

\section{Feature Selection}\label{sec:featureselection}
\section{Dataset Splitting}\label{sec:datasetsplitting}
After completing the Feature Selection phase (Section
\ref{sec:featureselection}), we decided to divide the obtained
dataset into three different sets:

%Completata anche la fase di Feature Selection (Sezione
%\ref{sec:featureselection}) abbiamo deciso di suddividere il dataset
%ottenuto in 3 differenti set:

\begin{itemize}
	\item \textit{Training}: It covers the period from 01-06-2022 to
	      28-02-2023. This set will be used during the training phase
	      to allow the model to learn the plant's performance and how
	      to predict energy trends during various gaps.
	      It starts in June because there are some issues with the
	      Inverter 1's features before that.

	      % comprende il periodo che va dal 01-06-2022
	      %     al 28-02-2023. Questo set sarà utilizzato durante la fase di Training
	      %     per permettere al modello di apprendere l'andamento dell'impianto e 
	      %     come riuscire a prevedere l'andamento dell'energia durante i vari buchi. Parte da giugno perchè prima ci sono alcuni problemi con le feature dell'Inverter 1.
	\item \textit{Validation}: It spans from 01-03-2023 to 31-03-2023.
	      It will be used in the training phase to assess the presence
	      of overfitting or if this phase might have failed.

	      % va dal 01-03-2023 al 31-03-2023. Verrà utilizzato nella fase di training per capire se c'è presenza di overfitting o se questa fase potrebbe aver fallito.
	\item \textit{Testing}: It starts on 01-04-2023 and continues
	      until 30-04-2023. This set will be used in the Evaluation phase
	      to estimate the model's performance on a data period it has
	      never seen before.

	      % parte dal 01-04-2023 fino ad arrivare al 30-04-2023. Questo set sarà utilizzato nella fase di Evaluation per
	      % poter stimare le performance del modello su un periodo di dati che non ha mai visto prima.
\end{itemize}

\begin{table}[H]
	\begin{center}
		\begin{tabular}[c]{l|l|l|l}
			%\cline{2-4}
			\multicolumn{1}{c|}{}               &
			\multicolumn{1}{c|}{\textbf{Start}} &
			\multicolumn{1}{c|}{\textbf{End}}   &
			\multicolumn{1}{c}{\textbf{Rows}}                                     \\
			\hline

			\textbf{Training}                   & 01-06-2022 & 28-02-2023 & 24864 \\
			\textbf{Validation}                 & 01-03-2023 & 31-03-2023 & 2880  \\
			\textbf{Testing}                    & 01-04-2023 & 30-04-2023 & 2880
		\end{tabular}
	\end{center}
	\caption{Summary table of how the dataset was divided.}\label{tab:dfsplit}
\end{table}
