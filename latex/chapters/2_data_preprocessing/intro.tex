Having the dataset divided into different files, one for each day
and device (see Section \ref{sec:dataset}), with the presence of
some periods, ranging from a few minutes to several days,
of missing data (likely due to a data collection tool failure),
results in many issues during the training phase and makes it
almost impossible. In this chapter, we will see how we have
addressed these problems by opting for a monolithic tabular
structure for the dataset and discussing the approaches we have
used for handling missing data.

%Avere il dataset suddiviso in diversi file, uno per giorno e per
%dispositivo (vedi Sezione \ref{sec:dataset}), con la presneza di
%alcuni periodi, che vanno da pochi minuti a diversi gionri, di assenza dati
%(dovuti probabilmente ad un fallimento dello strumento di raccolta dati)
%fa sì che si verifichino molti problemi durante la fase di training
%e la rende quasi impossibile. In questo capitolo vedremo come abbiamo
%risolto questi problemi, optando per una struttura tabellare monolitica 
%del dataset e quali approccia abbiamo utilizzato per la gestione
%dei dati mancanti.