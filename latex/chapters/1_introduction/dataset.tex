\section{Dataset}

Il dataset a nostra disposizione descrive un periodo di quasi due anni (dal
02/02/2022 al 16/06/2023) ed è proveniente da un impianto fotovoltaico da 978 kW
situato nella provincia di Bari. È composto da
3 inverter, 27 quadri di campo, 1 contatore, 1 solarimetro, 2 protezioni
interfaccia e 1 dispositivo \textquote{impianto} nel quale vengono immagazzinati
i dati provenienti da Solargis. Il dataset è organizzato in file, uno per
tipologia di dispositivo, che
rappresentano la singola giornata e i dati sono aggregati a 5 minuti ed ogni
riga riporta il riferimento al
dispositivo di appartenenza (\verb|deviceName| e \verb|deviceId|).

\begin{table}[H]
	\begin{center}
		\begin{tabular}[c]{l}
			\hline
			\textbf{File Name}                                  \\
			\hline
			\verb|2022_02_02_Rofilo_NP00003174_inverter.csv|    \\
			\verb|2022_02_02_Rofilo_NP00003174_meteorology.csv| \\
			\verb|2022_02_02_Rofilo_NP00003174_meter.csv|       \\
			\verb|2022_02_02_Rofilo_NP00003174_other.csv|       \\
			\verb|2022_02_02_Rofilo_NP00003174_plantDevice.csv| \\
			\verb|2022_02_02_Rofilo_NP00003174_stringbox.csv|   \\
			\verb|2022_02_03_Rofilo_NP00003174_inverter.csv|    \\
			\verb|2022_02_03_Rofilo_NP00003174_meteorology.csv| \\
			\verb|2022_02_03_Rofilo_NP00003174_meter.csv|       \\
			\verb|2022_02_03_Rofilo_NP00003174_other.csv|       \\
			\verb|2022_02_03_Rofilo_NP00003174_plantDevice.csv| \\
			\verb|2022_02_03_Rofilo_NP00003174_stringbox.csv|   \\
			$\ldots$                                            \\
			\hline
		\end{tabular}
	\end{center}
	\caption{Estratto di alcuni file del dataset.}\label{tab:datasunto}
\end{table}

%% TODO: migliorare dati
\begin{table}[H]
	\begin{center}
		\begin{tabular}[c]{l|l|l|l|l|l}
			\hline
			\multicolumn{1}{c|}{\textbf{timestamp}}        &
			\multicolumn{1}{c|}{\textbf{serial}}           &
			\multicolumn{1}{c|}{\textbf{$\ldots$}}         &
			\multicolumn{1}{c|}{\textbf{TotalEnergy(kWh)}} &
			\multicolumn{1}{c|}{\textbf{Frequency(Hz)}}    &
			\multicolumn{1}{c}{\textbf{deviceid}}            \\
			\hline
			01/02/2023 00:05                               &
			INV01                                          &
			$\ldots$                                       &
			431324.36                                      &
			49.88                                          &
			83204                                            \\

			01/02/2023 00:10                               &
			INV01                                          &
			$\ldots$                                       &
			431324.36                                      &
			49.88                                          &
			83204                                            \\

			01/02/2023 00:15                               &
			INV01                                          &
			$\ldots$                                       &
			431324.36                                      &
			49.88                                          &
			83204                                            \\

			$\ldots$                                       &
			$\ldots$                                       &
			$\ldots$                                       &
			$\ldots$                                       &
			$\ldots$                                       &
			$\ldots$                                         \\
			01/02/2023 12:00                               &
			INV01                                          &
			$\ldots$                                       &
			431324.36                                      &
			49.88                                          &
			83204                                            \\
			01/02/2023 12:05                               &
			INV01                                          &
			$\ldots$                                       &
			431324.36                                      &
			49.88                                          &
			83204                                            \\
			01/02/2023 12:10                               &
			INV01                                          &
			$\ldots$                                       &
			431324.36                                      &
			49.88                                          &
			83204                                            \\

			$\ldots$                                       &
			$\ldots$                                       &
			$\ldots$                                       &
			$\ldots$                                       &
			$\ldots$                                       &
			$\ldots$                                         \\
			01/02/2023 23:45                               &
			INV01                                          &
			$\ldots$                                       &
			431324.36                                      &
			49.88                                          &
			83204                                            \\
			01/02/2023 23:50                               &
			INV01                                          &
			$\ldots$                                       &
			431324.36                                      &
			49.88                                          &
			83204                                            \\
			01/02/2023 23:55                               &
			INV01                                          &
			$\ldots$                                       &
			431324.36                                      &
			49.88                                          &
			83204                                            \\
			\hline
		\end{tabular}
		\caption{Contenuto di un file che rappresenta un inverter}\label{tab:invsunto}
	\end{center}
\end{table}

\subsection{Inverter}
Un inverter in un impianto fotovoltaico è un componente fondamentale che svolge
una funzione vitale: converte l'energia elettrica prodotta dai pannelli solari,
che è in corrente continua (DC), in energia elettrica utilizzabile in corrente
alternata (AC). Questa conversione è cruciale perché la maggior parte delle
apparecchiature elettriche domestiche e delle reti di distribuzione elettrica
utilizzano corrente alternata per funzionare.
Regolano la
tensione e la frequenza della corrente alternata prodotta per garantire che
siano conformi agli standard di rete elettrica locale. Questo è importante
per evitare danni agli apparecchi elettrici e per garantire
l'interoperabilità con la rete elettrica.
Sono spesso dotati di
tecnologie avanzate come il "Maximum Power Point Tracking" (MPPT) che
ottimizzano costantemente la produzione di energia solare. Questo significa
che l'inverter regola la tensione in ingresso dai pannelli solari per
ottenere la massima potenza possibile in base alle condizioni di luce solare
in tempo reale.
Le feature che caratterizzano il nostro impianto sono riassunte dalla seguente
tabella:

\begin{table}[H]
	\begin{center}
		\begin{tabular}[c]{l|l|l}
			\hline
			\multicolumn{1}{c|}{\textbf{Name}}        &
			\multicolumn{1}{c|}{\textbf{Unit Symbol}} &
			\multicolumn{1}{c}{\textbf{Description}}                                         \\
			\hline
			CommunicationCode                         & -   & COMMUNICATION CODE             \\
			Failure 3                                 & -   & Allarme Attivo                 \\
			Failure 4                                 & -   & Allarme di isolamento          \\
			CurrentDC                                 & A   & Corrente di campo fotovoltaico \\
			CurrentAC                                 & A   & Corrente di Rete               \\
			CurrentAC Phase1                          & A   & Corrente RMS di Linea Fase R   \\
			CurrentAC Phase2                          & A   & Corrente RMS di Linea Fase S   \\
			CurrentAC Phase3                          & A   & Corrente RMS di Linea Fase T   \\
			TotalEnergy                               & kWh & Energia Attiva Erogata         \\
			Frequency                                 & Hz  & Frequenza di Rete              \\
			PowerAC Phase1                            & kW  & PA di Linea Fase R             \\
			PowerAC Phase2                            & kW  & PA di Linea Fase S             \\
			PowerAC Phase3                            & kW  & PA di Linea Fase T             \\
			PowerAC                                   & kW  & PA Erogata                     \\
			PowerDC                                   & kW  & Potenza di campo fotovoltaico  \\
			Status                                    & -   & Stato Inverter                 \\
			Failure                                   & -   & Stato PLL per Aggancio Rete    \\
			Failure 2                                 & -   & Stato Rete 1                   \\
			Failure 1                                 & -   & Stato Rete 2                   \\
			InternalTemperature                       & C   & Temp. CPU                      \\
			HeatSinkTemperature                       & C   & Temp. IGBT                     \\
			VoltageDC                                 & V   & Tensione di campo fotovoltaico \\
			VoltageAC                                 & V   & Tensione di Rete               \\
			VoltageAC Phase1                          & V   & Tensione RMS di Linea Fase R   \\
			VoltageAC Phase2                          & V   & Tensione RMS di Linea Fase S   \\
			VoltageAC Phase3                          & V   & Tensione RMS di Linea Fase T   \\
			\hline
		\end{tabular}
		\caption{Lista di tutte le features degli inverter, con descrizione.}\label{tab:invfeatures}
	\end{center}
\end{table}

%%TODO: aggiungere grafico inverter TotalEnergy, Temperature, e altri ?

\subsection{StringBox}
Le stringbox in un impianto fotovoltaico sono contenitori elettrici progettati
per ospitare e proteggere una serie (o stringa) di pannelli solari collegati in
serie. Questi dispositivi svolgono diverse funzioni importanti:

\begin{itemize}
	\item Protezione: Le stringbox includono dispositivi di protezione come interruttori
	      automatici o fusibili che prevengono cortocircuiti e sovraccarichi nel circuito
	      fotovoltaico.
	\item Connettori: Forniscono connettori sicuri per collegare i cavi provenienti dai
	      pannelli solari alla stringa di cavi principale dell'impianto.
	\item Monitoraggio: Alcune stringbox sono dotate di sistemi di monitoraggio che
	      consentono di rilevare prestazioni o guasti dei pannelli solari all'interno
	      della stringa.
	\item Isolamento: Possono anche includere dispositivi di isolamento che consentono di
	      interrompere l'alimentazione elettrica verso la stringa di pannelli solari per
	      scopi di manutenzione o sicurezza.
\end{itemize}

\subsection{JunctionBox}

\begin{table}[H]
	\begin{center}
		\begin{tabular}[c]{l|l|l}
			\hline
			\multicolumn{1}{c|}{\textbf{Name}}        &
			\multicolumn{1}{c|}{\textbf{Unit Symbol}} &
			\multicolumn{1}{c}{\textbf{Description}}                                           \\
			\hline
			CommunicationCode                         & -              & COMMUNICATION CODE    \\
			Failure                                   & -              & Allarme Stringhe      \\
			CurrentString1                            & A              & Corrente I1           \\
			CurrentString2                            & A              & Corrente I2           \\
			CurrentString3                            & A              & Corrente I3           \\
			CurrentString4                            & A              & Corrente I4           \\
			CurrentString5                            & A              & Corrente I5           \\
			CurrentString6                            & A              & Corrente I6           \\
			CurrentString7                            & A              & Corrente I7           \\
			AverageStringCurrent                      & A              & Corrente Media        \\
			Irradiance                                & $\text{W/m}^2$ & Irraggiamento moduli  \\
			Failure 1                                 & -              & Stringhe Aperte       \\
			Failure 2                                 & -              & Stringhe Non Perform. \\
			EnvironmentTemperature                    & C              & Temperatura ambiente  \\
			ModuleTemperature                         & C              & Temperatura moduli    \\
			InternalTemperature                       & C              & Temperatura Scheda    \\
			\hline
		\end{tabular}
		\caption{Lista di tutte le features deglle JunctionBox, con descrizione.}\label{tab:junctionfeatures}
	\end{center}
\end{table}


\subsection{Solargis}
Solargis è una società specializzata nella fornitura di dati e servizi di
previsione solare per impianti fotovoltaici e progetti legati all'energia
solare. Il loro principale obiettivo è fornire informazioni precise e affidabili
sull'irradiazione solare e sulle condizioni meteorologiche solari in qualsiasi
parte del mondo. Questi dati sono fondamentali per la progettazione,
l'ottimizzazione e la gestione degli impianti fotovoltaici.
Solargis raccoglie e fornisce dati dettagliati
sull'irradiazione solare globale, diretta e diffusa in ogni posizione
geografica. Questi dati consentono agli sviluppatori di impianti fotovoltaici di
valutare la quantità di energia solare disponibile in una determinata area, il
che è fondamentale per dimensionare correttamente l'impianto e calcolare le
previsioni di produzione.

\begin{table}[H]
	\begin{center}
		\begin{tabular}[c]{l|l|l|l}
			\hline
			\multicolumn{1}{c|}{\textbf{timestamp}}         &
			\multicolumn{1}{c|}{\textbf{$\ldots$}}          &
			\multicolumn{1}{c|}{\textbf{SolargisGHI(W/m2)}} &
			\multicolumn{1}{c}{\textbf{SolargisGTI(W/m2)}}                          \\
			\hline
			2022-08-01 11:40:00                             & $\ldots$ & 896 & 978  \\
			2022-08-01 11:45:00                             & $\ldots$ & 896 & 978  \\
			2022-08-01 11:50:00                             & $\ldots$ & 896 & 978  \\
			2022-08-01 11:55:00                             & $\ldots$ & 914 & 1001 \\
			2022-08-01 12:00:00                             & $\ldots$ & 914 & 1001 \\
			2022-08-01 12:05:00                             & $\ldots$ & 914 & 1001 \\
			2022-08-01 12:10:00                             & $\ldots$ & 928 & 1019 \\
			2022-08-01 12:15:00                             & $\ldots$ & 928 & 1019 \\
			2022-08-01 12:20:00                             & $\ldots$ & 928 & 1019 \\

			\hline
		\end{tabular}
		\caption{Dati di Solargis a nostra disposizione, provenienti dal file
			\texttt{2022\_08\_01\_Rofilo\_NP00003174\_plantDevice.csv}}\label{tab:solargis}
	\end{center}
\end{table}

%%TODO: aggiungere grafico Solargis

Solar radiation takes a long journey until it reaches Earth’s surface. So when
modelling solar radiation, various interactions of extra-terrestrial solar
radiation with the Earth’s atmosphere, surface and objects are to be taken into
account. The component that is neither reflected nor scattered, and which
directly
reaches the surface, is called direct radiation; this is the component that
produces shadows. The component that is scattered by the atmosphere, and which
reaches the ground is called diffuse radiation. The small part of the radiation
reflected by the surface and reaching an inclined plane is called the reflected
radiation. These three components together create global radiation.

In solar energy applications, the following parameters are commonly used in
practice:

\begin{itemize}
	\item Direct Normal Irradiation/Irradiance (DNI) is the component that is
	      involved in thermal (concentrating solar power, CSP) and photovoltaic
	      concentration
	      technology (concentrated photovoltaic, CPV).
	\item  Global Horizontal
	      Irradiation/Irradiance (GHI) is the sum of direct and diffuse radiation
	      received on a horizontal plane. GHI is a reference radiation for the
	      comparison of climatic zones; it is also essential parameter for
	      calculation of radiation on a tilted plane.
	\item Global Tilted Irradiation/Irradiance (GTI), or total
	      radiation received on a surface with defined tilt and azimuth, fixed or
	      sun-tracking. This is the sum of the scattered radiation, direct and
	      reflected. It is a reference for photovoltaic (PV) applications, and
	      can be occasionally affected by shadow.
\end{itemize}


\begin{table}[H]
	\begin{center}
		\begin{tabular}[c]{l|l|l}
			\hline
			\multicolumn{1}{c|}{\textbf{Name}}        &
			\multicolumn{1}{c|}{\textbf{Unit Symbol}} &
			\multicolumn{1}{c}{\textbf{Description}}                                                            \\
			\hline
			SolargisGHI                               & $\text{W/m}^2$ & Solargis Global Horizontal Irradiation \\
			SolargisGTI                               & $\text{W/m}^2$ & Solargis Global Tilted Irradiation     \\
			\hline
		\end{tabular}
		\caption{Lista di tutte le features di Solargis, con descrizione.}\label{tab:solargisfeatures}
	\end{center}
\end{table}

