\section{Problem Definition}
La seguente tesi si pone l'obbiettivo di risolvere il problema dell'imputazione
di serie temporali di dati provenienti da impianti fotovoltaici. Nello
specifico, accade molto spesso che gli strumenti di acquisizione dati di un
impianto falliscano momentaneamente causando un periodo di tempo, più o meno
lungo, dove la curva dell'energia totale prodotta è assente. Per cercare di
colmare questo "buco" non basta un semplice calcolo dato che vanno tenuti in
considerazione vari fattori come l'irraggiamento solare, temperatura ambientale,
presenza di nuvole, pioggia, ecc.
Formalmente possiamo definire il problema come segue:

\begin{definition}
    Siano dati:
    \begin{enumerate}
        \item  Un insieme di serie temporali di dati rappresentanti un impianto
              fotovoltaico, $S = \{S_1, S_2, ..., S_N\}$, dove $N$ rappresenta
              il numero di serie temporali disponibili;
        \item Una serie temporale target $t \in S$ che rappresenta l'Energia
              Totale Prodotta;
        \item Ogni serie temporale $S_i$ è composta da coppie ordinate $(t_i, v_i)$,
              dove $t_i$ è un timestamp che rappresenta il momento in cui è stato
              registrato il valore $v_i$.
    \end{enumerate}

    L'obiettivo del problema dell'imputazione è stimare i valori mancanti o
    danneggiati nella serie temporale $t$.
\end{definition}

% Data una serie temporale proveniente da un impianto fotovoltaico, definiamo come
% \textit{buco} l'assenza di dati per un periodo limitato di tempo e la
% \textit{target feature} come la feature che vogliamo predirre (nel caso
% specifico l'Energia Totale Prodotta). L'obbiettivo è quindi quello di restituire
% in output l'andamento della \textit{target feature} durante il periodo di
% \textit{buco} facendo attenzione che la quantità dell'energia prodotta alla fine
% del \textit{buco} sia esattamente uguale a quella nota immediatamente dopo.






