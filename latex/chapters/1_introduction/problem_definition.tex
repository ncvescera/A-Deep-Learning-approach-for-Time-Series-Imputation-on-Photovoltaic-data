\section{Problem Definition}
This thesis aims to address the problem of imputing time series data from
photovoltaic systems. Specifically, it often happens that data acquisition
instruments in a system temporarily fail, causing a period of time, more or
less extended, where the curve of the total energy produced is missing.
%{\bf (*VP* la frase seguente non va bene per una tesi, non è scientifica, cosa sarebbe una simple formula? Direi semplicemente quali sono le cose da considerare e la variabilità dettata anche dai diversi momenti della giornata e/o dell'anno )}
Closing these gaps is a highly complex task because we must consider not only various factors such as solar irradiance, ambient temperature, cloud cover, rainfall, etc., but also the changing seasons, the passage of hours, and the day/night cycle.
%To try to fill this \textquote{gap} a simple formula is not sufficient,
%as various factors such as solar irradiance, ambient temperature,
%presence of clouds, rainfall, etc., need to be taken into account.

Formally, we can define the problem as follows. %\textbf{(*VP* nella def l'attenzione è puntata sui given invece che sull'oggetto della formula.)}

\begin{definition}[\textbf{Imputation problem}]
	%Sia T una serie temporale con n osservazioni, indicata come T = {(t_1, y_1), (t_2, y_2), ..., (t_n, y_n)}, dove t_i rappresenta il tempo e y_i è il valore osservato in quel momento.
	%
	%Il problema dell'imputazione dei dati provenienti da serie temporali può essere formulato come segue:
	%
	%Dato un sottoinsieme S di indici i = {i_1, i_2, ..., i_m}, dove 1 ≤ i_j ≤ n, il compito è stimare i valori mancanti y_i per i_j appartenenti a S, utilizzando le informazioni dai punti temporalmente adiacenti.

	Given a set of time series data representing a photovoltaic system, $S = {S_1, S_2, \ldots, S_N}$, where $N$ represents the number of available time series;
	a target time series $t \in S$ that represent the Total Generated Energy, where each time series $S_i$ is composed of ordered pairs $(t_i, v_i)$, where $t_i$ is
	a timestamp representing the moment when the value $v_i$ was recorded.

	The objective of the imputation problem is to estimate the missing values $v_j'$ for $t_j'$ belonging to $T'$, using information from temporally adjacent time points, where $T'$ is a subset of indices $\{t_1', t_2', ..., t_m'\} \in t$, where $1 \le t_j' \le n$.

\end{definition}
%La seguente tesi si pone l'obbiettivo di risolvere il problema dell'imputazione
%di serie temporali di dati provenienti da impianti fotovoltaici. Nello
%specifico, accade molto spesso che gli strumenti di acquisizione dati di un
%impianto falliscano momentaneamente causando un periodo di tempo, più o meno
%lungo, dove la curva dell'energia totale prodotta è assente. Per cercare di
%colmare questo \textquote{buco} non basta un semplice calcolo dato che vanno tenuti in
%considerazione vari fattori come l'irraggiamento solare, temperatura ambientale,
%presenza di nuvole, pioggia, ecc.
%Formalmente possiamo definire il problema come segue:

%\begin{definition}
%    Siano dati:
%    \begin{enumerate}
%        \item  Un insieme di serie temporali di dati rappresentanti un impianto
%              fotovoltaico, $S = \{S_1, S_2, ..., S_N\}$, dove $N$ rappresenta
%              il numero di serie temporali disponibili;
%        \item Una serie temporale target $t \in S$ che rappresenta l'Energia
%              Totale Prodotta;
%        \item Ogni serie temporale $S_i$ è composta da coppie ordinate $(t_i, v_i)$,
%              dove $t_i$ è un timestamp che rappresenta il momento in cui è stato
%              registrato il valore $v_i$.
%    \end{enumerate}
%
%    L'obiettivo del problema dell'imputazione è stimare i valori mancanti o
%    danneggiati nella serie temporale $t$.
%\end{definition}
%
% Data una serie temporale proveniente da un impianto fotovoltaico, definiamo come
% \textit{buco} l'assenza di dati per un periodo limitato di tempo e la
% \textit{target feature} come la feature che vogliamo predirre (nel caso
% specifico l'Energia Totale Prodotta). L'obbiettivo è quindi quello di restituire
% in output l'andamento della \textit{target feature} durante il periodo di
% \textit{buco} facendo attenzione che la quantità dell'energia prodotta alla fine
% del \textit{buco} sia esattamente uguale a quella nota immediatamente dopo.






